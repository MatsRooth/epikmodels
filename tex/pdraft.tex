%
% File acl2020.tex
%
%% Based on the style files for ACL 2020, which were
%% Based on the style files for ACL 2018, NAACL 2018/19, which were
%% Based on the style files for ACL-2015, with some improvements
%%  taken from the NAACL-2016 style
%% Based on the style files for ACL-2014, which were, in turn,
%% based on ACL-2013, ACL-2012, ACL-2011, ACL-2010, ACL-IJCNLP-2009,
%% EACL-2009, IJCNLP-2008...
%% Based on the style files for EACL 2006 by 
%%e.agirre@ehu.es or Sergi.Balari@uab.es
%% and that of ACL 08 by Joakim Nivre and Noah Smith

\documentclass[11pt,a4paper]{article}
\usepackage[hyperref]{acl2020}
\usepackage{times}
\usepackage{latexsym}
\usepackage{enumitem}
\usepackage{mathtools}
%\usepackage{caption}
\renewcommand{\UrlFont}{\ttfamily\small}

\newcommand{\ehc}[1]{\textit{\textcolor{red}{[ehc]: #1}}}
\newcommand{\mr}[1]{\textit{\textcolor{cyan}{[mr]: #1}}}
% \newcommand{\ehc}[1]{}
% \newcommand{\mr}[1]{}


\newcommand{\K}{\mathcal{K}}
\newcommand{\B}{\mathcal{B}}

%\captionsetup{belowskip=0pt}

%\setlength{\abovecaptionskip}{-10pt}
\setlength{\belowcaptionskip}{-15pt}

\setlength{\abovecaptionskip}{5pt plus 3pt minus 2pt}

% This is not strictly necessary, and may be commented out,
% but it will improve the layout of the manuscript,
% and will typically save some space.
\usepackage{microtype}

% MR249 UNCOMMENTED FOR FINAL VERSION
\aclfinalcopy % Uncomment this line for the final submission
%\def\aclpaperid{***} %  Enter the acl Paper ID here

%\setlength\titlebox{5cm}
% You can expand the titlebox if you need extra space
% to show all the authors. Please do not make the titlebox
% smaller than 5cm (the original size); we will check this
% in the camera-ready version and ask you to change it back.

\newcommand\BibTeX{B\textsc{ib}\TeX}

\title{Epistemic Semantics in Guarded String Models\thanks{Thanks to Tobias Kapp\'{e} for insightful reactions to the paper.
Thanks also to three SCiL reviewers for helpful comments.
Earlier versions of the work were presented in the LUSH talk series at Utrecht in December 2019, and in the Workshop on formal/philosophical/computational approaches to natural language meaning at Rochester in December 2020.  Thanks to the audiences for their reactions.}}
\renewcommand\footnotemark{}
%%\renewcommand\footnoterule{}

\author{Eric Hayden Campbell \\
  Cornell University\\
  \texttt{ehc86@cornell.edu} \\\And
  Mats Rooth \\
  Cornell University\\
  \texttt{mr249@cornell.edu} \\}

\date{}

\usepackage{amssymb,amsmath,accents}
\usepackage{stmaryrd}
\SetSymbolFont{stmry}{bold}{U}{stmry}{m}{n}
\usepackage{array}
\usepackage{hang}
\renewcommand{\UrlFont}{\ttfamily\small}
\usepackage{microtype}

\newcommand{\dom}{\ensuremath{\mathrm{dom}}}
\newcommand{\subheading}[1]{\noindent {\bf \large #1}}
%%\raisebox{-.6ex}{\rm \tiny \scriptsize ordinary}}
%labeled syntax brackets
\newcommand{\npi}[2]{[\raisebox{-0.6ex}{\rm \tiny NP }#1]\raisebox{-0.6ex}{\tiny #2}}
\newcommand{\dpi}[2]{[\raisebox{-0.6ex}{\rm \tiny DP }#1]\raisebox{-0.6ex}{\tiny #2}}
\newcommand{\xpi}[2]{[\raisebox{-0.6ex}{\rm \tiny XP }#1]\raisebox{-0.6ex}{\tiny #2}}
\newcommand{\xp}[1]{[\raisebox{-0.6ex}{\rm \tiny XP }#1]}
\newcommand{\si}[2]{[\raisebox{-0.6ex}{\rm \tiny S
}#1]\raisebox{-0.6ex}{\tiny #2}}
\newcommand{\vpi}[2]{[\raisebox{-0.6ex}{\rm \tiny VP }#1]\raisebox{-0.6ex}{\tiny #2}}
\newcommand{\np}[1]{[\raisebox{-0.6ex}{\rm \tiny NP}#1]}
\newcommand{\detp}[1]{[\raisebox{-0.6ex}{\rm \tiny DP}#1]}
\newcommand{\phr}[1]{[\raisebox{-0.6ex}{\rm \tiny ph}#1]}
\newcommand{\s}[1]{[\raisebox{-0.6ex}{\rm \tiny S }#1]}
\newcommand{\lb}[2]{[\raisebox{-0.6ex}{\rm \tiny #1 }#2]}
\newcommand{\ip}[1]{[\raisebox{-0.6ex}{\rm \tiny IP }#1]}
\newcommand{\discourse}[1]{[\raisebox{-0.6ex}{\rm \tiny D }#1]}
\newcommand{\n}[1]{[\raisebox{-0.6ex}{\rm \tiny N }#1]}
\newcommand{\vv}[1]{[\raisebox{-0.6ex}{\rm \tiny V }#1]}
\newcommand{\p}[1]{[\raisebox{-0.6ex}{\rm \tiny P }#1]}
\newcommand{\adj}[1]{[\raisebox{-0.6ex}{\rm \tiny A }#1]}
\newcommand{\nbar}[1]{[\raisebox{-0.6ex}{\rm \tiny \nbarsymbol }#1]}
\newcommand{\vbar}[1]{[\raisebox{-0.6ex}{\rm \tiny \vbarsymbol }#1]}
\newcommand{\ibar}[1]{[\raisebox{-0.6ex}{\rm \tiny \ibarsymbol }#1]}
\newcommand{\sbar}[1]{[\raisebox{-0.6ex}{{\rm \tiny{S'}}}#1]}
\newcommand{\vp}[1]{[\raisebox{-0.6ex}{\rm \tiny VP}#1]}
\newcommand{\pp}[1]{[\raisebox{-0.6ex}{\rm \tiny PP }#1]}
\newcommand{\ei}[1]{e\raisebox{-0.6ex}{\tiny #1}}
\newcommand{\tv}[1]{[\raisebox{-0.6ex}{\rm \tiny TV}#1]}
\newcommand{\scriptP}[0]{{\cal P}}
\newcommand{\pr}[0]{\bf \rm}
\newcommand{\subF}[0]{\raisebox{-0.6ex}{\tiny F}}
\newcommand{\sub}[1]{\raisebox{-0.6ex}{{\tiny{#1}}}}
\newcommand{\up}[1]{\raisebox{1.2ex}{{\tiny{#1}}}}
%temporal indices for Enc stuff  FIX THESE !!!!!!!!!!!! -- get rid of $
\newcommand{\COMPi}[1]{COMP_{#1}}
\newcommand{\PSTi}[1]{PST$_{#1}}
\newcommand{\PRSi}[1]{PRS$_{#1}  }
%right margin remarks for drafts
\newcommand{\remark}[1]{\marginpar{\tiny \raggedright #1}}
%lambda
\newcommand{\lamb}[2]{$\lambda$#1#2}
\newcommand{\Lamb}[2]{$\Lambda$#1#2}
%examples
\newcounter{exampleno}
\newenvironment{example}%
{\refstepcounter{exampleno} 
 \begin{list}{}{\setlength{\leftmargin}{0.45in}
                \setlength{\topsep}{0.1in}
                \setlength{\partopsep}{0.0in}
                \setlength{\itemsep}{0.0in}
                \setlength{\labelwidth}{0.3in}
                \setlength{\parsep}{0.0in}
                \setlength{\labelsep}{0.1in}}
 \item[(\theexampleno)\hfill]}%
{\end{list}}
\newenvironment{exampletree}%
{\refstepcounter{exampleno} 
 \begin{list}{}{\setlength{\leftmargin}{0.0in}
                \setlength{\topsep}{0.1in}
                \setlength{\partopsep}{0.0in}
                \setlength{\itemsep}{0.0in}
                \setlength{\labelwidth}{0.0in}
                \setlength{\parsep}{0.0in}
                \setlength{\labelsep}{0.1in}}
 \item[(\theexampleno)\hfill]}%
{\end{list}}
\newenvironment{examples}%
{\refstepcounter{exampleno} 
 \begin{list}{}{\setlength{\leftmargin}{0.55in}
                \setlength{\topsep}{0.1in}
                \setlength{\partopsep}{0.0in}
                \setlength{\itemsep}{0.0in}
                \setlength{\labelwidth}{0.4in}
                \setlength{\parsep}{0.0in}
                \setlength{\labelsep}{0.1in}}}%
{\end{list}}
\newcommand{\itema}{ \item[(\theexampleno){\hfill}a.]}
\newcommand{\itemb}{ \item[b.]}
\newcommand{\itemc}{ \item[c.]}
\newcommand{\itemd}{ \item[d.]}
\newcommand{\iteme}{ \item[e.]}
\newcommand{\itemf}{ \item[f.]}
\newcommand{\itemg}{ \item[g.]}
\newcommand{\itemh}{ \item[h.]}
\newcommand{\ignore}[1]{}
\newenvironment{sexp}{\ignore\{}{\}}
% stuff for association with focus paper
\newcommand{\focus}[1]{#1\raisebox{-.6ex}{\rm \scriptsize F}}
\newcommand{\F}{\mbox{\raisebox{-.6ex}{{\rm\scriptsize{F}}}}}
\newcommand{\T}{\mbox{\raisebox{-.6ex}{{\rm\scriptsize{T}}}}}
\newcommand{\SOF}{\mbox{\raisebox{-.6ex}{{\rm\scriptsize{SOF}}}} }
\newcommand{\down}{\mbox{\raisebox{.6ex}{$\scriptsize{\vee}$}}}
%\newcommand{\semval}[1]{[\![#1]\!]\raisebox{0.8ex}{\rm \scriptsize ordinary}}
\newcommand{\semval}[1]{\mbox{[\hspace{-0.4ex}[}#1\mbox{]\hspace{-0.4ex}]}\raisebox{.8ex}{\rm \scriptsize o}}
\newcommand{\semvaluf}[1]{\mbox{[\hspace{-0.4ex}[}#1\mbox{]\hspace{-0.4ex}]}\raisebox{.8ex}{\scriptsize U,F}}
\newcommand{\sem}[1]{\llbracket#1\rrbracket} %{\mbox{[\hspace{-0.4ex}[}#1\mbox{]\hspace{-0.4ex}]}}
\newcommand{\focussem}[1]{\mbox{[\hspace{-0.4ex}[}#1\mbox{]\hspace{-0.4ex}]}\raisebox{.8ex}{\rm \scriptsize f}}
\newcommand{\semfocus}[1]{\mbox{[\hspace{-0.4ex}[}#1\mbox{]\hspace{-0.4ex}]}\raisebox{.8ex}{\rm\scriptsize f}}
\newcommand{\semtopic}[1]{\mbox{[\hspace{-0.4ex}[}#1\mbox{]\hspace{-0.4ex}]}\raisebox{.8ex}{\rm\scriptsize t}}
\newcommand{\semalt}[1]{\mbox{[\hspace{-0.4ex}[}#1\mbox{]\hspace{-0.4ex}]}\raisebox{.8ex}{\rm\scriptsize a}}
\newcommand{\semord}[1]{\mbox{[\hspace{-0.4ex}[}#1\mbox{]\hspace{-0.4ex}]}\raisebox{.8ex}{\rm\scriptsize o}}
\newcommand{\nat}[0]{I\hspace{-0.7ex}N}
\newcommand{\real}[0]{I\hspace{-0.7ex}R}
\newcommand{\setabs}[2]{\left\{#1\middle|#2\right\}}
\newcommand{\pair}[2]{\left\langle#1,#2\right\rangle}
\newcommand{\triple}[3]{\left\langle#1,#2,#3\right\rangle}
\newcommand{\set}[1]{\left\{#1\right\}}
\newcommand{\tuple}[1]{\langle#1\rangle}
\newcommand{\powerset}[1]{{\cal P}(#1)}
\newcommand{\thereis}[2]{{\exists}#1\left[#2\right]}
\newcommand{\some}[2]{{\exists}#1\left[#2\right]}
\newcommand{\all}[2]{{\forall}#1\left[#2\right]}
\newcommand{\element}[2]{#1 \in #2 }
\newcommand{\ifthen}[2]{\left[#1 \rightarrow #2\right]}
\newcommand{\Prime}[1]{{\rm\bf{#1}}}
\newcommand{\const}[1]{{\rm\bf#1}}
\newcommand{\eqdef}[0]{{\stackrel{\rm def}{=}}}
\newcommand{\StarPrime}[1]{\raisebox{-.6ex}{*}{\rm\bf#1}}
\newcommand{\lam}[2]{\lambda#1\left[#2\right]}
\newcommand{\lami}[2]{\lambda^{I}#1\left[#2\right]}
%\mbox{$[\![$}
%\mbox{$]\!]$}
%%\newcommand{\tag}[1]{\label{#1} \marginpar{\raggedright \em #1}}
%%\newcommand{\tag}[1]{\label{#1}}
\newcommand{\exi}[2]{(\ref{#1}#2)} 
\newcommand{\ex}[1]{(\ref{#1})} 
\newcommand{\nbarsymbol}[0]{{{{\rm{N'}}}}}
\newcommand{\vbarsymbol}[0]{{{{\rm{V'}}}}}
\newcommand{\ibarsymbol}[0]{{I'}}
\newcommand{\sbarsymbol}[0]{\mbox{\({\rm S}'\)}}

\newcommand{\adjp}[1]{[\raisebox{-0.6ex}{\rm\tiny{ADJP}}#1]}
\newcommand{\adv}[1]{[\raisebox{-0.6ex}{\rm\tiny{ADV}}#1]}
\newcommand{\aux}[1]{[\raisebox{-0.6ex}{\rm\tiny{AUX}}#1]}
\newcommand{\conjp}[1]{[\raisebox{-0.6ex}{\rm\tiny{CONJP}}#1]}
\newcommand{\conj}[1]{[\raisebox{-0.6ex}{\rm\tiny{CONJ}}#1]}
\newcommand{\dart}[1]{[\raisebox{-0.6ex}{\rm\tiny{DART}}#1]}
\newcommand{\fin}[1]{[\raisebox{-0.6ex}{\rm\tiny{FIN}}#1]}
\newcommand{\unattached}[1]{[\raisebox{-0.6ex}{\rm\tiny{?}}#1]}
\newcommand{\iart}[1]{[\raisebox{-0.6ex}{\rm\tiny{IART}}#1]}
\newcommand{\npl}[1]{[\raisebox{-0.6ex}{\rm\tiny{NPL}}#1]}
\newcommand{\pnp}[1]{[\raisebox{-0.6ex}{\rm\tiny{PNP}}#1]}
\newcommand{\prep}[1]{[\raisebox{-0.6ex}{\rm\tiny{P }}#1]}
\newcommand{\pro}[1]{[\raisebox{-0.6ex}{\rm\tiny{PRO }}#1]}
\newcommand{\tns}[1]{[\raisebox{-0.6ex}{\rm\tiny{TNS }}#1]}
\newcommand{\ving}[1]{[\raisebox{-0.6ex}{\rm\tiny{VING }}#1]}
\newcommand{\vpprt}[1]{[\raisebox{-0.6ex}{\rm\tiny{VPPRT }}#1]}
\newcommand{\vpres}[1]{[\raisebox{-0.6ex}{\rm\tiny{VPRES }}#1]}
\newcommand{\vpast}[1]{[\raisebox{-0.6ex}{\rm\tiny{VPAST }}#1]}

\newcommand{\pw}[2]{{\sc#1}{---#2}}

\newcommand{\ldq}[0]{\mbox{``}}
\newcommand{\rdq}[0]{\mbox{''}}
\newcommand{\notelement}[2]{\mbox{$#1\,\hspace{-1ex}\not
    \makebox[-1pt]{}\epsilon\,#2$}}

\newcommand{\wordtone}[2]{
\begin{tabular}[t]{c}#1\\#2\end{tabular}
}

\newcommand{\boundary}{{\noindent...\hrulefill.....\hrulefill...}}
\newcounter{Exampleno}
\newenvironment{Example}[1]%
{\vspace{18pt}\par\refstepcounter{Exampleno}\par\noindent%
{\bf Example \theExampleno\hspace{0.5cm}#1\par}}{\qed\par\vspace{18pt}}

%%% Local Variables: 
%%% mode: latex
%%% TeX-master: t
%%% End: 

\usepackage{wasysym,graphicx}
\usepackage{tcolorbox}
\usepackage{booktabs}

\newcommand{\event}{\sf{P}}
\newcommand{\test}{\sf{B}}

\begin{document}
\maketitle
\begin{abstract}
  Constructive and computable multi-agent epistemic possible worlds models are
  interpreted as sets of guarded string models in an epistemic extension of
  Kleene Algebra with Tests (\textsc{Kat}). The account is framed as a formal
  language Epi\textsc{Kat} (Epistemic \textsc{Kat}) for defining such models.
  The language is implemented by translation into the finite state calculus, and
  alternatively by modeling propositions as lazy lists in Haskell. The
  syntax-semantics interface for a fragment of English is defined by a
  categorial grammar.
\end{abstract}


\nocite{beesley2003finite,hulden2009foma}
\nocite{karttunen2010update}

\section{Introduction and Related Work}
Linguistic semantics in the Montague tradition proceeds by assigning
propositional {\em semantic values} to disambiguated sentences of a natural
language. A proposition is a set or class of {\em possible worlds}, which are
often assumed to have the same nature and complexity as the world we occupy
\citep{lewis1986plurality}. But alternatively, one can work with small idealized
models, to illustrate and test ideas. The point of this paper is to extend
standard idealized models to countable sets of worlds, and to constructively and
computably model alternatives for epistemic agents.

Enter Epi\textsc{Kat}, which is a systematic way of defining such models, and can be
applied to natural language semantics, specifically, epistemic semantics and
clausal embedding. The fundamental insight is to identify possible worlds with
strings of primitive events, so that propositions are sets of (guarded) strings,
whose regular sets have a rich algebraic characterization~\cite{kozen1997kleene}
and computational model~\cite{kozen1997kleenecompleteness,kozen2001automata}.
Epi\textsc{Kat} leverages these algebras to define a mathematical model
(Sections~\ref{sec:model} \&~\ref{sec:logic}) and computationally interpret
using finite state machines (Section~\ref{sec:fst}) and lazy lists of strings
(Section~\ref{sec:lazy})).

{\em Related Work.} Epi\textsc{Kat} synthesizes five antecedents. John
McCarthy's {\em Situation Calculus} is the source of the idea of constructing
possible worlds as event sequences
\citep{mccarthy1963situations,reiter2001knowledge}. The algebraic theory of {\em
  Kleene Algebra with Tests} (\textsc{Kat}) characterizes algebras of regular
sets of guarded strings \citep{kozen2001automata}, which form the basis for
Epi\textsc{Kat}'s propositions and event types. {\em Action models} in dynamic
epistemic semantics introduce the technique of constructing epistemic models
from primitive alternative relations on events, in order to capture the
epistemic consequences of perceptual and communicative events
\citep{baltag1999logic}. Literature on {\em finite state methods in linguistic
  semantics} has used event strings and sets of event strings to theorize about
tense and aspect in natural language semantics
\citep{fernando2004finite,fernando2007observing,carlson2009tense} and to express
intensions \cite{fernando2017intensions}. Work on {\em finite state intensional
  semantics} has investigated how to do the semantics of intensional
complementation in a setting where compositional semantics is expressed in a
finite state calculus \citep{rooth2017finite,collard2018finite}.


{\em Example.} Consider an example event-sequence model called {\em The
  Concealed Coin}. Amy and Bob are seated at a table. There is a coin on the
table under a cup. The coin could be heads-up ($H$) or tails-up ($T$), and
neither agent knows which it is. Call this initial possible world $w_1$.
Additional worlds can be constructed via the events in \ex{coin1}.

\newcommand{\nop}{\ensuremath{o}}

\newcommand{\apH}{\ensuremath{a_h}}
\newcommand{\apT}{\ensuremath{a_t}}
\newcommand{\bpH}{\ensuremath{b_h}}
\newcommand{\bpT}{\ensuremath{b_t}}
\newcommand{\cpH}{\ensuremath{c_h}}
\newcommand{\cpT}{\ensuremath{c_t}}

\newcommand{\atTH}{\ensuremath{a_{th}}}
\newcommand{\atHT}{\ensuremath{a_{ht}}}
\newcommand{\btTH}{\ensuremath{b_{th}}}
\newcommand{\btHT}{\ensuremath{b_{ht}}}


\begin{example}  \label{coin1} \begin{enumerate}[align=left,leftmargin=0.75em]
  \item[\apH] Amy peeks at $H$, by tipping the cup. Bob sees she's peeking, but not what she sees.
  \item[\bpH] Bob peeks at $H$.
  \item[\apT] Amy peeks at $T$.
  \item[\bpT] Bob peeks at $T$. 
  \item[\atTH] Amy secretly turns the coin from $T$ to $H$.
    She knows she turned the coin over, but not which side was face up. Bob thinks nothing happened.
  \item[\atHT] Amy secretly turns the coin from $H$ to $T$.
  \item[\btTH] Bob secretly turns the coin from $T$ to $H$.
  \item[\btHT] Bob secretly turns the coin from $H$ to $T$.
\end{enumerate}
\end{example}

The worlds in \ex{coin2} are examples constructed from the events in \ex{coin1}.
Juxtaposition indicates the order of events; for example, $w_3$ indicates that
after the events of $w_2$, event $b_1$ occurs, which, starting in the initial
world $w_1$ corresponds to Amy peeking at a heads-up coin, and then Bob doing
the same.

\begin{example} \label{coin2}
  \hspace{-1.5em}
  \(w_2 \triangleq w_1\apH \hfill
  w_3 \triangleq w_2\bpH \hfill
  w_4 \triangleq w_3\atHT\btTH\bpH\)
\end{example}

Now, the truth ($1$) or falsity ($0$) of English sentences can be evaluated in
each world $w_i$, i.e. \emph{after} the events have happened. See \ex{coin3} for
examples such as in $w_4$, where Bob knows it's heads (because he peeked after
turning the coin), but Amy does not (because Bob secretly turned the coin over).

\begin{example} \label{coin3}
\hspace{-7mm} \begin{tabular}[t]{ccccl}
  $w_1$ & $w_2$  & $w_3$ & $w_4$ & Sentence\\
  $0$ & $1$ & $1$& $0$ & Amy knows it's heads. \\
  $0$ & $0$ & $1$& $1$ & Bob knows it's heads. \\
  $0$ & $0$ & $1$& $0$ & Bob knows Amy \\ &&&& knows it's heads. \\
  $0$ & $1$ & $1$& $0$ & Bob knows Amy \\ &&&& knows  whether it's \\ &&&& heads or tails. \\
\end{tabular}
\end{example}

%\ehc{I'm not sure if these sentences are the most interesting anymore!}

\newcommand{\ev}[2]{#1 \mathbin{:} #2}
\newcommand{\evr}[2]{\ev{#2}{#1}}

To create realistic models, contradictory event sequences must be prohibited.
For example, $\apH\atTH$ would require the coin to simultaneously be heads-up
and tails-up. To mechanize such reasoning, events in Epi\textsc{Kat} come with
Boolean pre-conditions and post-conditions \emph{\`a la} Hoare
Logic~\cite{hoare1969axiomatic}. In the running example, the Boolean variables
$h$ (the coin is heads-up) and $t$ (the coin is tails-up) represent the coin's
state. Then, since Amy can only observe heads when the coin is heads-up, event
$\apH$ has the invariant condition $h$. Similarly, Amy can only turn the coin
from tails to heads if it shows tails, so $\atTH$ has the tails-up precondition
$t$ and the heads-up postcondition $h$.

Pre- and post- conditions are expressed using an operator ``$:$'' (read ``and
next'') that pairs Boolean formulas into an \emph{effect formula}. Specifically,
the effects for $\apH$ are written $\ev h h$, and for $\atTH$ are written $\ev t
h$. Effect formulas are interpreted as relations on boolean valuations, as defined
in Figure~\ref{effectfigure}.

However, a coin cannot simultanously show both heads and tails! Currently, the
precondition $h$ of $\atHT$ only says that heads must be showing, and says
nothing about the fact that tails must be face-down, indicated by the formula
$\bar t$. The effect valuations are restricted to only the feasible ones via a
\emph{state formula}, shown, and demonstrated for $\apH$ in \ref{effect}. The
state formula says that the coin is either heads-up or tails-up, but not
both\footnote{Two Booleans constrained by a state formula, is more illustrative
  a single boolean $h$ where tails is just $\bar h$.}. Here juxtaposition
represents conjuction, $+$ is disjunction, and overbar is negation.

With the state formula in hand, $\apH\atTH$ is evidently contradictory, since
the post-condition $h$ of $\apH$ is incompatible with the pre-condition $t$ of
$\atTH$. See Figure~\ref{effectfigure} for more details.


%\renewcommand{\arraystretch}{1.0}

Given a set {\sf{B}} of state primitives and $\varphi$ a state formula over
{\sf{B}}, define ${\cal{A}}_{\mbox{\sf\tiny{B}}}^{\varphi}$ to be the set of
valuations of {\sf{B}} that make formula $\varphi$ true. The valuations for the
coin example are shown in \ref{effect}, using the sequence notation for
valuations, e.g. $\bar{h}{t}$, wherein every state primitive is listed in fixed
order, and left unmarked (indicating true) or marked with the overbar
(indicating false). Valuations are called atoms, because they correspond to the
atoms of a Boolean algebra of tests \citep{kozen2001automata}.

\begin{example}   \label{effect}
\(\begin{array}[t]{ll}
{\sf{B}} & \{h,t\}\\
\mbox{state formula $\varphi$} & h\bar{t} + \bar{h}t \\
\mbox{effect formula $\zeta_{\apH}$ for $\apH$} &  h \mathbin{:} h \\
 \mathcal{A}^\varphi_{\mathsf B} & \{h\bar{t},\bar{h}t\}   \\
\sem{\zeta_{\apH}}^\varphi & \{\pair{h\bar{t}}{h\bar{t}}\}   \\
\end{array}\)
\end{example}

Thus far, the presentation is closely related to Kleene Algebra with
Hypotheses~\cite{cohen1994hypotheses}, which permits user-specification of
futher equations beyond the axioms of \textsc{Ka}. Specifically, the state
formula $\varphi$ can be thought of as a hypothesis $\varphi \equiv 1$, and an
event formula $\zeta_a$ for event $a$, is simply the hypothesis $a \equiv
\sum_{\varphi,\varphi' \in \llbracket \zeta_a \rrbracket} \varphi;a;\varphi'$.
Rather than rely on general techniques ~\cite{doumane2019kleene,
  hardin2002elimination, kozen1997kleenecompleteness}, we realize our specific
hypotheses directly.

Epi\textsc{Kat} is first and foremost a multi-agent epistemic logic, with
epistemic operators that cannot naively be represented as
hypotheses\footnote{Characterizing this relationship precisely is future work.}.
Aside from formalizing the metaphysical modality seen so far,
Sections~\ref{sec:model} and~\ref{sec:logic} develop the epistemic modality.


\begin{figure}
  \fbox{
    \begin{minipage}{0.96\columnwidth}
  \(\begin{array}{lcl}
  \multicolumn 3 l {\textrm{state formulas}\hfill (a \in \mathsf{B}) \hfill \mbox{}}\\
  \rho,\sigma,\varphi &::=& a \mid0 \mid 1 \mid\rho+\sigma \mid \rho\,\sigma \mid \bar{\rho} \\
  \multicolumn {3} l {\textrm{effect formulas}}\\
  \zeta,\eta & ::= & \rho \mathbin{:} \sigma \mid \zeta+\eta
    \mid \zeta \mathbin{\&} \eta \mid \bar{\zeta}
  \end{array}\)
  
  \(
  \begin{array}{lll}
    \sem{\rho\mathbin{:}\sigma}^\varphi & \triangleq &
        {\mathcal A}^{\rho\varphi}_{\mathsf{B}}\times{\mathcal A}^{\sigma\varphi}_{\mathsf{B}}\\[.1em]
        \sem{\zeta +\eta}^\varphi & \triangleq & \sem{\zeta}^\varphi\cup\sem{\eta}^\varphi\\
        \sem{\zeta \mathbin{\&}  \eta}^\varphi & \triangleq & \sem{\zeta}^\varphi\cap\sem{\eta}^\varphi\\
        \sem{\bar{\zeta}}^\varphi & \triangleq & {\mathcal A}^{\varphi}_{\mathsf{B}} \times {\mathcal A}^{\varphi}_{\mathsf{B}} \setminus \sem{\zeta}^\varphi\\
  \end{array}  
  \)
  \end{minipage}}
\caption{Syntax of state formulas and syntax and semantics of effect formulas. Effect formulas
  denote relations between atoms. In a state formula, juxtaposition $\rho\,\sigma$ is conjunction.}
\label{effectfigure}
\end{figure}

\vspace{1mm}


\section{Epistemic guarded string models}
\label{sec:model}

Epi\textsc{Kat} is a specification language for possible worlds models that
includes declarations of events and states, state formulas, effect
formulas, and additional information.  Figure \ref{fig:epik-example} shows
an Epi\textsc{Kat} program that describes a possible worlds model for two agents
with information about one coin, events of the agents semi-privately
looking at the coin, and events of secretly turning the coin.  The
line beginning with {\tt state} enumerates ${\sf{B}}$. The line
beginning with {\tt restrict} gives the state formula.  The lines
beginning with {\tt event} declare events and their effect formulas.

Finally, the lines beginning with {\tt agent} define {\em event alternative}
relations for the epistemic agents in the model. Each clause with an arrow has a
single event symbol on the left, and a disjunction of alternative events on the
right of the arrow. The interpretation of Amy's alternatives for $\bpH$ (Bob
peeks at heads), is that when $\bpH$ happens, for Amy either $\bpH$ or $\bpT$ (Bob
peeks at tails) could be happening, indicating that she doesn't know whether Bob saw heads or tails, only that he peeked at the coin.

Similarly, her alternatives for $\atTH$ (she turns the coin over from heads to
tails) are $\atTH$ and $\atHT$, indicating that she doesn't know, \emph{a
  priori}, whether she's turning the coin from $H$ to $T$ or from $T$ to $H$.
She has the same alternatives for the event $\atHT$. Conversely, when Bob
secretly turns the coin over, in event $\btTH$ or $\btHT$, she doesn't know
anything has happened, so her alternative is the ``no-operation'' or
``no-information'' event $\nop$. Bob's event relation is symmetric.

\begin{figure}
  \begin{tcolorbox}
    \footnotesize
    \begin{minipage}{0.48\textwidth}
      \begin{center}    
        {\tt
          \begin{tabular}{l}
            state h t \\
            restrict  h!t \\
            \phantom{restrict }+ t!h \\
            event o  h:h + t:t\\
            event a1  h:h \\
            event a0  t:t \\
            event b1  h:h \\
            event b0  t:t \\
            event a10 h:t \\
            event a01 t:h \\
            event b10 h:t \\
            event b01 t:h \\
      \end{tabular}}
      \end{center}
    \end{minipage} \hfill 
    \begin{minipage}{0.51\textwidth}
      \begin{center}
        {\tt
          \begin{tabular}{l}
            agent amy \\
            \hspace{3mm} o  -> o \\
            \hspace{3mm} a1  -> a1 \\
            \hspace{3mm} a0  -> a0 \\
            \hspace{3mm} b1  -> b1 + b0\\
            \hspace{3mm} b0  -> b1 + b0\\
            \hspace{3mm} a10 -> a10 + a01 \\
            \hspace{3mm} a01 -> a10 + a01 \\
            \hspace{3mm} b10 -> o \\
            \hspace{3mm} b01 -> o \\
            agent bob \\
            \hspace{3mm} <\textrm{\textit{sim.} swap }a\textrm{ and }b>
            %% agent bob \\
            %% \hspace{3mm} b1 -> b1 \\
            %% \hspace{3mm} b0 -> b0 \\
            %% \hspace{3mm} a1 -> a1 + a0\\
            %% \hspace{3mm} a0 -> a1 + a0\\
        \end{tabular} }
      \end{center}
    \end{minipage}
  \end{tcolorbox}
  
\caption{Epi\textsc{Kat} program describing a possible-worlds event sequence
  model for two agents with information about one coin, and events of
  the agents semi-privately looking at the coin, and privately turning
  the coin.}
\label{fig:epik-example}
\end{figure}


The sequel focuses on defining a concrete possible worlds model from an Epi\textsc{Kat}
specification. The models are an extension of guarded-string models for Kleene
Algebra with Tests (\textsc{Kat}). This algebraic theory has model classes including
guarded string models, relational models, finite models, and matrix models. Our
definitions and notation follow \cite{kozen2001automata}. We add syntax and
semantics to cover multi-agent epistemic semantics.

Guarded strings over a finite alphabet ${\sf{P}}$ are like ordinary strings, but
with atoms over a set ${\sf{B}}$ alternating with the symbols from ${\sf{P}}$.
In the model described by Figure~\ref{fig:epik-example}, ${\sf{P}}$ is the set of
events $\{\apH,\apT,\bpH,\bpT,\atTH,\atHT,\btTH,\btHT,\nop\}$, and ${\sf{B}}$ is
$\{h,t\}$. As we already saw in \ex{effect}, $\mathcal{A}^\varphi_{\mathsf B}$
is $\{h\bar{t},\bar{h}t\}$, for which we use the shorthand $\{H,T\}$. A guarded
string over ${\sf{P}}$ and ${\sf{B}}$ is a string of events from ${\sf{P}}$,
alternating with atoms over ${\sf{B}}$, and beginning and ending with atoms. In
this construction, $w_1=H$, $w_2=H \apH H$, $w_3=H \apH H \bpH H$, and
$w_4=H \apH H \bpH H \atHT T \btTH H \bpH H$.

The discussion of \ex{coin2} mentioned building worlds by incrementing smaller
worlds with events and maintaining the pre- and post- conditions. This is
accomplished in guarded string models with the fusion product ($x \diamond y$),
a partial operation that combines two guarded strings $x$ and $y$, subject to
the condition that the atom at the end of the $x$ is identical to the atom at
the start of $y$. \ex{fusion1} gives some examples.


\begin{example} \label{fusion1}
%\hspace{-2mm}\(\hat 0 \,u\, \hat 1 \,u\, \hat 2 \,d\, \hat 1 \; \diamond \; \hat 1 \,u\, \hat 2 = \hat 0 \,u\, \hat 1 \,u\, \hat 2 \,d\, \hat 1 \,u\, \hat 2 \)

%\hspace{-2mm}\( \hat 0 \,u\, \hat 1 \,u\, \hat 2 \,d\, \hat 1 \; \diamond \; \hat 2 \,u\, \hat 3 = \textit{undefined} \)

\hspace{-2mm}\(H\,\bpH\,H \; \diamond \; H \,\apH\, H = H\, \bpH\, H \,\apH\, H \)

\hspace{-2mm}\(T\,\btTH\,H \; \diamond \; T \,\apH\, T = \textit{undefined} \)
\end{example}

Rather than individual guarded strings, elements of a guarded string model for
\textsc{Kat} are sets of guarded strings. In Epi\textsc{Kat}, these elements
have the interpretation of propositions, which are (regular) sets of possible
worlds. In a free guarded string model for \textsc{Kat}, any event can be
adjacent to any atom in a guarded string that is an element of the underlying
set for the algebra. However, this subsumes the valid worlds defined by the
state and effect formulas. \ex{wf1} defines the \emph{well-formed} guarded
strings (valid worlds) determined by an Epi\textsc{Kat} specification. Condition
(i) says that each atom is consistent with the state constraint, and condition
(ii) says that each constituent token event $\alpha_ie_i\alpha_{i+1}$ is
consistent with its effects\footnote{ An alternative is to
  define equations such as $\bar{\phi}=0$ (from the state formula $\phi$) and
  $\apH\,{=}\,h \apH h$ (from the effect formula $h\mathbin{:}h$ for event
  $\apH$), and construct a quotient algebra from the equivalence relation
  generated by these equations. This results in equating sets of guarded strings
  in the free algebra that differ by guarded strings that are ill-formed
  according to the state and effect formulas. In the development in the text, we
  instead use a set of guarded strings that are well-formed according to the
  state and effect formulas as the representative of the equivalence class.}.

\begin{example}   \label{wf1}
Given $\sf{P}$, $\sf{B}$, a
state formula $\varphi$, and an effect formula $\zeta_e$ for each
event $e$ in $\sf{P}$,    
$\alpha_0e_0 ... e_n\alpha_{n+1}$ is well-formed iff

\vspace{1mm}

\hspace{2mm}(i) $\element{\alpha_i}{{\mathcal A}^{\varphi}_{\mathsf{B}}}$ ($0{\leq}i{\leq}n$), and 

\hspace{2mm}(ii) $\element{\tuple{\alpha_i,\alpha_{i+1}}}{\sem{\zeta_{e_i}}^{\varphi}}$, ($0{\leq} i {\le} n$).
\end{example}

Well-formed guarded strings have the interpretation of worlds in the application
to natural-language semantics. The set of possible worlds in the modal frame
determined by an Epi\textsc{Kat} specification is the set of well-formed guarded
strings, and propositions are regular sets of guarded strings, rather than the
power set \citep{montague1975formal,gallin1975intensional}. Certain sets of
well-formed guarded strings have the addional interpretation of event types. An
event-type is something that can ``happen'' in different worlds. For example,
$\apH$ corresponds to the event type $\{H\,\apH\,H\}$, and $\nop$ to the event
type $\{T\,\nop\,T, H\,\nop\,H\}$.

The construction so far defines a set of worlds from an Epi\textsc{Kat} specification.  Normally
the set is countably infinite, though some choices of effect formulas can result in a
finite set of worlds.  The next step is to define an alternative relation $R_a$ on
worlds for each agent $a$. This will result in a general modal frame
$\tuple{W,R_1,...,R_n,K}$ consisting of a set of worlds, a world-alternative relation
for each agent, and a set $K$ of propositions, where each proposition is a subset of
$W$ \citep{chagrov1997modal}.\footnote{As explained in the next section, $K$ will not
  be the power set of $W$, rather it will consists of the regular subsets of $W$.}

An Epi\textsc{Kat} specification defines an alternative relation on bare events for each
agent $a$, which is notated ${R}_{a}$, and lifted to a relation $\hat{R}_a$ on
worlds. The basic idea is that when a world $w$ is incremented with an event
$e$, in the resulting world $w \diamond e$, epistemic alternatives for agent $a$
are of the form $w' \diamond e'$, where $w'$ is an alternative to for $a$ in
$w$, and $e'$ is and event-alternative to $e$ for $a$.\footnote{In this it is
  important that the event-alternative relation for an agent is constant across
  worlds. We anticipate that the definition given here produces results
  equivalent to what is found in literature on event alternatives in dynamic
  epistemic semantics, though we have not verified this. That literature
  primarily focuses on mapping an epistemic model for a single time and
  situation to another, and uses general first-order models, rather than guarded
  string models. See \citet{baltag1999logic}, \citet{van2007dynamic}, and
  articles in \citet{ditmarsch2015handbook}. This literature is motivated by
  epistemic logic and AI planning, rather than computable possible worlds models
  in natural language semantics. } This needs to be implemented in a way that
takes account of pre- and post-conditions for events. For this, our approach is
to refer the definition of well-formed guarded strings. \ex{raise1} defines an
epistemic alternative relation on worlds from an alternative relation on bare
events.

\begin{example}  \label{raise1}
  Let $W$ be a set of guarded strings over events P and primitive
  tests B, and ${R}$ be a relation on $P$. The corresponding
  relation $\hat{R}$ on $W$ holds between a guarded string
\(
\alpha_0e_0 ... e_n\alpha_{n+1} 
\)
in $W$ and a guarded string  $q$ in $W$ iff $q$ is of the form
\(
\alpha_0'e_0' ... e_n'\alpha_{n+1}', 
\), where for $0 \leq n$, $e_iRe'_i$.
\end{example}  

This requires that in an alternative world, each constituent event $e'_i$ is an
alternative to the corresponding event $e_i$ in the base world.
Compatibilities between events in the alternative world are enforced by the
requirement that the alternative world is an element of $W$, so that state and effect
formulas are enforced.

Consider a scenario like the one from Figure~\ref{fig:epik-example}, augmented
with an additional agent Cal. The base world $T\,\bpT\,T\,\cpT\,T$ is one where
the coin is tails, and first Bob looks at tails, and them Cal looks at tails.
The first event $\bpT$ has the alternatives $\bpT$ and $\bpH$ for Amy, and the
second event $\bpT$ has the alternatives $\cpT$ and $\cpH$ for Amy. This results
in four combinations $\bpT\cpT$, $\bpT\cpH$, $\bpH\cpT$, and $\bpH\cpH$, some of
which, like $\bpT\cpH$, are bogus. These contradictions are filtered by pre-
and post-conditions of events in the alternative world, so that the set of
alternatives for Amy in $T\,\bpT\,T\,\cpT\,T$ is $\{T\,\bpT\,T\,\cpT\,T,
H\,\bpH\,H\,\cpH\,H\}$, with two world-alternatives instead of four.


\begin{figure}
  \fbox{
    \begin{minipage}{0.96\columnwidth}
      \(\begin{array}{rlrl}
      \multicolumn{4}{l}{\text{events}~~e\in \mathsf{P}} \\
      p,q &\multicolumn{3}{l}{::= e\,|\,\sigma\,|\,p+q\,|\,pq\,|\,p^*\,|\,\neg p\,|\, \Diamond_ap} \\
      \Box_a p & \triangleq \neg\Diamond_a\neg p &
      \bullet & \triangleq \sum_{e \in \mathsf P}e \\      
      p \wedge q &  \triangleq \neg(\neg p + \neg q) &
      p \to q & \triangleq \neg p + q
      \end{array} \)
  \end{minipage}}

\caption{The language of Epi\textsc{Kat} terms and key derived
  operators.}
\label{fig:epik-grammar}
\label{termfigure}
\end{figure}


\section{The logical language of Epistemic \textsc{Kat}}
\label{sec:logic}

The standard language for Kleene algebra with tests has the signature
$\tuple{K,+,\cdot,*,\bar{},0,1}$ \citep{kozen2001automata}.  In a guarded string
model for \textsc{Kat}, $K$ is a set of sets of guarded strings, $+$ is set union, the
operation $\cdot$ is fusion product raised to sets, $*$ is Kleene star, the operation
$\;\bar{}\;$ is complement for tests, $0$ is the empty set, and $1$ is the set of
atoms.\footnote{$0$ has the dual role the identity for $+$ (union), and as False for
  operations on tests.  $1$ has the dual role of the identity for product (fusion
  product raised to sets), and True for tests.  } To this we add a unary
modal operation $\Diamond_a$ for each agent, and a unary complement operation $\neg$
on elements of $K$.  Intuitively, $\Diamond_a p$ is the set of
worlds where proposition $p$ is epistemically possible for agent $a$.  Propositional
complement is included because natural languages have sentence negation.  In
addition, universal box modalities are defined as duals of existential diamond
modalities.

With modalities and propositional negation added, the signature of $n$-agent
Epi\textsc{Kat} is $\tuple{K,+,\cdot,*,\bar{},0,1,\neg,\Diamond_1,
  ..., \Diamond_n}$. Figure \ref{termfigure} defines the syntax of the language.
Juxtaposition is used for product.  Terms in this language
are used to represent the propositional semantic values of English sentences.

Consider the examples in \ex{interpretation1}. The first term, $\bullet^*\,h$ uses
the syntactic sugar $\bullet$, defined in Figure \ref{fig:epik-grammar} to be
the disjunction of the primitive events. Since a world is a well-formed sequence
of events, $\bullet^*$ is the set of worlds. Multiplying by the state symbol $h$
in the term $\bullet^* h$ has the effect of conjoining $h$ with the atom at the
end of the world. So $\bullet^* h$ is the set of worlds where the coin ends
heads-up.

\begin{example}   \label{interpretation1}
\hspace{-1.5em}\begin{tabular}[t]{ll}
${\bullet}^*t$ & It's tails. \\
$\bullet^*h$ & It's heads. \\  
  $ \bullet^* h \wedge \Box_a \bullet^* h$  & Amy knows it's heads. \\
\end{tabular}
\vspace{1mm}
\hspace{-1.5em}\begin{tabular}[t]{l}
  $\Box_b(\bullet^*t \wedge \Box_a\bullet^*t + \neg\bullet^*t \wedge \Box_a\neg\bullet^*t)$\\
 Bob believes Amy knows whether it's tails.  \\
\end{tabular}
\end{example} 

{\em Standard Epistemic Modalities.} Using the modal primitive $\Diamond_a$, and the
dual encodings of $\Box_a$ and $\wedge$, we can encode the standard modal
operators expressing knowledge $(\K_a)$ and belief $(\B_a)$ as
\ex{encodings}.\footnote{Deeper analysis of the lexical semantics of {\em know}
  requires adding modeling of presupposition \citep{collard2018finite}. The
  grammar fragment in Section 6 does not model the presupposition of {\em know},
  except as an entailment.}


\begin{example}\label{encodings}
  \( \begin{array}[t]{ccl}
    \textsc{Belief} & \B_a~p &\triangleq \Box_a~p \\
    \textsc{Knowledge} & \K_a~p & \triangleq p \wedge \B_a~p
    \end{array}\)
\end{example}

%% As a consequence of these encodings, the standard dualities
%% hold. These are shown in \ex{modalprops}.
%% \begin{example} \label{modalprops}
%%   \(\K_a~p \to \B_a~p \qquad \B_a~p \to \neg\K_a\neg p \)
%% \end{example}

Different types of reasoners (e.g. accurate, inaccurate,
etc) are modeled using the event-alternatives in an
Epi\textsc{Kat} specification.\footnote{See the discussion of modal axioms \textbf{T} and
  \textbf{D} below.}  The agents in Figure 1 do not always have reliable beliefs,
because of the possibility of secret turning.


\newcommand{\epikspec}{\test,\varphi,\zeta}
\newcommand{\AphiB}{\mathcal A^\varphi_{\mathsf B}}

{\em Guarded String Interpretation.} A term $p$ of the logical language is interpreted as a set of guarded
strings $\sem{p}^{\epikspec}$, where superscript captures dependence
on an Epi\textsc{Kat} specification.  Figure \ref{fig:interp2} defines the
interpretation.  The interpretation $\sem{ 1
}^{\epikspec}$ of the multiplicative identity 1 is the set of atoms that
satisfy the state constraint $\varphi$.  Where $b$ is a primitive
Boolean, $\sem{ b }^{\epikspec}$ is the set of atoms that satisfy the
state constraint and where $b$ is true.  Where $e$ is a primitive
event, $\sem{ e }^{\epikspec}$ is the set of guarded strings that have
the form of $e$ flanked by compatible atoms, as determined by the
event formula $\zeta_e$. The product $pq$ is
interpreted with fusion product raised to sets of guarded strings.
Kleene star is interpreted as the union of exponents ($p^n$ is
the $n$-times product of $p$ with itself, with $p^0 = 1$).
Propositional complement is complement relative to the set of worlds.
The epistemic formula $\Diamond_a p$ is interpreted with Kripke
semantics for epistemic modality, as the pre-image of $p$ 
under the world-alternative relation $\hat{R}_a$.

\newlength{\dhatheight}
\newcommand{\doublehat}[1]{%
    \settoheight{\dhatheight}{\ensuremath{\hat{#1}}}%
    \addtolength{\dhatheight}{-0.35ex}%
    \hat{\vphantom{\rule{1pt}{\dhatheight}}%
      \smash{\hat{#1}}}}




\begin{figure}
  \fbox{\begin{minipage}{0.97\columnwidth}
  \centering
  \(\begin{array}{l >{\triangleq}c l}
  \sem{ 0 }^{\epikspec} && \emptyset \\  
  \sem{ 1 }^{\epikspec} && \mathcal A^\varphi_{\mathsf B}\\
  \sem{ b }^{\epikspec} && \mathcal A^{b\varphi}_{\mathsf B}\\
  \sem{ \bar{\sigma} }^{\epikspec} && \AphiB \setminus \sem{\sigma}^{\epikspec}\\
  \sem{ e }^{\epikspec}&&  \setabs{\alpha e \beta}{\alpha \mathrel{\sem{\zeta_e}^\varphi} \beta} \\
  \sem{{p+q}}^{\epikspec} && \sem{{p}}^{\epikspec} \cup \sem{ q }^{\epikspec}\\
  \sem{{pq}}^{\epikspec} && \setabs{x \diamond y}{ \begin{array}{l}x \in \sem{p}^{\epikspec} \\ y \in \sem{q}^{\epikspec}\\\mbox{$x \diamond y$ is defined}\end{array}} \\
  \sem{p^*}^{\epikspec} && \displaystyle\bigcup_{n \geq 0} {\sem{ p^n }^{\epikspec}} \\
  \sem{ \neg p }^{\epikspec} && \sem{\bullet^*}^{\epikspec} \setminus \sem{ p }^{\epikspec} \\
  \end{array}
  \)
  
  \hspace{2mm}\(\sem{ \Diamond_a\,p } \hspace{4mm}{\triangleq} \hspace{4mm}\{x \mid \exists y . x{\hat{R}_a}y \wedge y \in \sem{ p }^{\epikspec}\} \)
  \end{minipage}}
  \caption{Interpretation of Epi\textsc{Kat} terms as sets of guarded strings}
  \label{fig:interp2}
\end{figure}

Summing up, given an Epi\textsc{Kat} specification $\epikspec$, term $p$ (as
defined syntactically in Figure \ref{termfigure}) is interpreted as a set of
guarded strings $\sem{p}^{\epikspec}$. Let $K^{\epikspec}$ be the sets that are
interpretations of terms. Then
$\tuple{K^{\epikspec},+,\cdot,*,\bar{},0,1,\neg,\Diamond_{a_1},...,\Diamond_{a_n}}$
is a concrete guarded string interpretation for the signature of
Epi\textsc{Kat}, with operations as in Figure \ref{fig:interp2}. This provides a
concrete $n$-agent general modal frame $\tuple{\sem{\bullet^*}^{\epikspec},
  \hat{R}_1,...,\hat{R}_n,K^{\epikspec}}$. The frame consists of a set of
worlds, an epistemic-alternative relation for each agent, and a set of
propositions, each of which is a set of worlds. It is used as a target for
natural-language interpretation in Section~\ref{sec:categorial-grammar}.



\label{sec:axioms}

{\em Axiomatic Classification}. To situate our logic as a modal logic, consider
the soundness of the standard modal axioms given our semantics
\citep{hughes1996new}. The axioms in \ex{valid-axioms} are valid.

\begin{example}\label{valid-axioms}
  \begin{enumerate}[align=left]
  \item[\textbf{N}] If $p$ is valid, then $\Box_ap$ is valid
  \item[\textbf{K}] $\Box_a(p \to q) \to \Box_a~p \to \Box_a~q$ is valid.
  \end{enumerate}
\end{example}

The axioms in \ex{conditionally-valid-axioms} are not valid, but do hold when
when the relation $R_a$ has a certain shape.

\begin{example}\label{conditionally-valid-axioms}
  \(\begin{array}[t]{lll}
  \textbf{T} & \Box_a p \to p & \text{if } g \mathrel{\hat{R}_a} g, \forall g \\
  \textbf{D} & \Box_a p \to \Diamond_a p & \text{if } \hat{R}_a \text{ is a function} \\
  \end{array}\)
\end{example}

The conditions on axioms \textbf{4}, \textbf{5} and \textbf{B} are just a
restatements of the theorem in relational terms.

\section{Translation into the finite state calculus}
\label{sec:fst}

The finite state calculus is an algebra of regular sets of strings and
regular relations between strings that was designed for use in
computational phonology and morphographemics
\citep{kaplan1994regular,beesley2003finite}.  Current implementations
allow for the definition of functions on regular sets and relations
\citep{hulden2009foma,linden2009hfst,karttunen2010update}.  Such definitions are used
here to construct a model for Epi\textsc{Kat} inside the finite state
calculus (Figures~\ref{fstfigure1}\&\ref{relclosure}).

The space of worlds is a set of ordinary (as opposed to guarded) strings.  Bit
sequences (sequences of $0$'s and $1$'s) encode atoms, and as before, these alternate
with event symbols to encode a world.  In this construction, $w_3$ of the example is
the string {\tt{1\!0\;ah\;1\!0\;bh\;1\!0}}.

\begin{figure}
  \begin{tcolorbox}[size=fbox]
\footnotesize
\begin{hangingpar}
  {\tt St} \hspace{2mm} \\
  Atoms such as $0110$.% The length is the number of generators.
\end{hangingpar}
  
\begin{hangingpar}
  {\tt UnequalStPair} \hspace{2mm} \\
  Sequence of two unequal atoms such as $0110~~0111$.
\end{hangingpar}

\begin{hangingpar}
\begin{verbatim}
define Wf0 ~[$ UnequalStPair];
\end{verbatim}
String that doesn't contain a non-matching pair of atoms.
\end{hangingpar}

\begin{hangingpar}
\begin{verbatim}
define Squash St -> 0 || St _;
\end{verbatim}
Rewrite relation deleting the second of two atoms.
\end{hangingpar}

\begin{hangingpar}
\begin{verbatim}
define Cn(X,Y) 
  [[[X Y] & Wf0] .o. Squash].l;
\end{verbatim}
\textsc{Kat} product.
\end{hangingpar}

\begin{hangingpar}
\begin{verbatim}
define Kpl(X) 
  [[[X+] & Wf0] .o. Squash].l;
\end{verbatim}
\end{hangingpar}

\begin{hangingpar}
\begin{verbatim}
define Kst(X) St | Kpl(X);
\end{verbatim}
\textsc{Kat} Kleene plus and Kleene star. The Fst operation {\tt |} is union.
\end{hangingpar}
\end{tcolorbox}
\caption{\raggedright Epi\textsc{Kat} product and star defined in Fst.}
\label{fstfigure1}
\end{figure}


Terms in the finite state calculus are interpreted as sets of strings,
or for relational terms, as relations between strings.
Computationally, the sets and relations are represented by finite
state acceptors.  As used here, a program in the Fst language of the
finite state calculus is a straight-line program that defines a
sequence of constants naming sets, constants naming relations, and
functions (defined as macros) mapping one or more regular sets or
relations to a regular set or relation. 

Translating the Epi\textsc{Kat} terms $0$, $1$, $b$, and $e$ is
straightforward: convert the atoms as previously described,
and decorate the events $e$ with their compatible atoms. For example, $a_1$ becomes
an Fst term denoting
$\{01\apH01\}$.

Further, Epi\textsc{Kat}'s union $(+)$ and intersection $(\wedge)$ operators can
be directly translated to Fst's built-in analogous operators ((\texttt{|}) and
(\texttt{\&}). Propositional complement in Epi\textsc{Kat} becomes Fst set
difference (\texttt{-}) from the set of all worlds.

Defining \textsc{Kat} product using Fst's set-lifted string concatenation (denoted by
juxtaposition $\texttt{X}~\texttt{Y}$) requires more care. Naively concatenating
strings with atoms (Boolean vectors) at both ends doubles atoms at the juncture, and
does not enforce the requisite atom equality.  To implement \textsc{Kat} product, we define
the binary operation \texttt{Cn}, which concatenates strings in the string algebra,
removes strings with non-matching atoms, and then deletes the second of two atoms to
create a set of well-formed guarded strings (Figure \ref{fstfigure1}). {\tt Wf0}
is the set of ordinary strings with only equal pairs of atoms, as
defined using Fst's containment operator \texttt{\$}. The \texttt{Squash} relation
uses Fst's rewrite notation to delete atoms (elements of \texttt{St}) that are
preceded by another atom\footnote{ This is a non-equal length regular relation. The
  finite state calculus includes such relations, and they can be used with relation
  composition and relation domain and co-domain.  They are restricted in that the
  complement and set difference for non-equal length relations is not defined.
  Epistemic alterative relations are equal-length relations.}. This relation is
applied via the composition (\texttt{.o.}) and codomain (\texttt{.l})
operators.
%% Deletion is accomplished by a
%% re-write rule in the finite state calculus, which is a notation for
%% defining regular relations by contextually constrained substitutions.
%% In this case, is {\tt Squash} is a regular relation that deletes an
%% atom (a sequence of 0's and 1's of a certain length) when it follows
%% an atom.

Kleene plus is defined in a similar way, using Kleene plus in the string
algebra, with checks for equality of atoms and deletion of atoms. Kleene star is
defined from Kleene plus the well-formed atoms \texttt{St}, which implements $1$.


\begin{figure}
  \begin{tcolorbox}[size=fbox]
  \footnotesize  
{\raggedright
  \begin{hangingpar}
%% {\tt{define RelKpl(R)}}\\
%% $\underbrace{\mbox{\tt{Squash.i\hspace{1pt}.o.}}}_c$
%% $\underbrace{\mbox{\tt Wf0\hspace{1pt}.o.}}_b$
%% $\underbrace{\mbox{\tt{[R+]}}}_a$
%% $\underbrace{\mbox{\tt .o.\hspace{1pt}Wf0}}_b$
%% $\underbrace{\mbox{\tt{.o.\hspace{1pt}Squash}}}_c$
{\tt{define RelKpl(R)}}\\
$\!\underbrace{\texttt{Squash.i}\,\texttt{.o.}}_c
\underbrace{\mbox{\tt Wf0\,.o.}}_b
\underbrace{\mbox{\tt{[R+]}}}_a
\underbrace{\mbox{\tt .o.\,Wf0}}_b
\underbrace{\mbox{\tt{.o.\,Squash}}}_c$

\end{hangingpar}}

\begin{tabular}{ll}
  $a$ & Relational Kleene plus in the string algebra \\
  $b$ & Constrain domain and co-domain to contain \\
      & no unmatched atoms. \\
  $c$ & Reduce doubled atoms to a single \\
      & atom in the domain and co-domain. \\
\end{tabular}

\begin{hangingpar}
{\tt{define RelKst(R)
    [St\hspace{1pt}.x.\hspace{1pt}St]\hspace{1pt}|\hspace{1pt}Kpl(X);}}\\ The Fst
operation {\tt{.x.}} is Cartesian product.  {\tt{R.i}} is the inverse of relation
{\tt{R}}.
\end{hangingpar}
\end{tcolorbox}
\caption{Definition in Fst of the Kleene concatenation closure of
  a relation between guarded strings.}
\label{relclosure}
\end{figure}


Finally, the epistemic alternative relation for each agent, i.e. the
\texttt{agent} relation from Figure~\ref{fig:epik-example} is implemented in Fst.
Combining the agent relation with the effect formula gives a relation on events
decorated with compatible atoms. Then the relation on worlds is constructed
using the closure of the concatenation product operation. The concatenation
product $R~S$ of two relations $R$ and $S$ is the set of pairs of the form
$\pair{x_1x_2}{y_1y_2}$, where $x_1 \mathrel R y_1$, and $x_2 \mathrel S y_2$.
In Fst, {\tt{R+}} is the closure of relation {\tt R} with respect to this
operation. Figure \ref{relclosure} defines the corresponding operation on sets
of guarded strings as encoded in Fst.\footnote{Relation concatenation in Fst
  differs from relation composition ({\tt{.o.}}), and the closure under
  discussion here is the closure of the former rather than the latter.} The
epistemic alternative relation on worlds for an agent is then defined as the
\textsc{Kat} relation concatenation closure {\tt RelKst} of the decorated-event
alternative relation for the agent.

%(Note that in both the string algebra and the guarded string algebra, the operation
%on relations is concatenation closure, not reflexive transitive closure.)


\section{Bounded Lazy Interpretation}
\label{sec:lazy}

Epi\textsc{Kat} specifications also target lazy lists in Haskell, rather than the direct
interpretation as sets. Using lists sidesteps checking the set invariant (that
elements are unique) for large sets, such as $\bullet^*$, and laziness allows us
to delay computing these large sets until they are actually needed. To sidestep
the infiteness of models, we parameterize the interpretation function on a
positive integer $n$ and only produce guarded strings of length $n$ or less.

%% We translate the Epi\textsc{Kat} terms into a Haskell algebraic datatype that
%% represents terms with the same signature as described in Section 2. We
%% then paramaterize the interpretation function on an integer $n$ that
%% describes the maximum length of a string we will produce.

The bounded interpretation into lists of strings is very similar to
the unbounded interpretation into sets of strings, except for the
(lazy) bounds checking. The full details are shown in
Figure~\ref{fig:bound-interp}. First note that when $n=0$, the
denotation is empty, written $[]$. Terms of the form $0$, $1$, $e$,
and $\psi$ have the same denotation as before, translated into a list
(written $\lfloor S\rfloor$, for a set $S$). We compute atoms using
BDDs, which concisely represent boolean functions~\cite{lee1959representation}.

We lift the remaining operators (except Kleene star) to their list equivalents:
union becomes list append (written $\texttt{+\!+}$); fusion product is lifted to
lists instead of sets, negation is implemented using list difference
($\setminus$), and the modal operator lifts the alternative relation over lists
of strings\footnote{Figure~7 depicts this using the list comprehension notation,
  which is analogous to set builder notation, except that it is written using
  square brackets. Element order is evoked by the keyword \texttt{for}, rather
  than using the unordered $\forall$.}. The only caveat to these direct
interpretations is that we lazily restrict the strings to have size $\leq n$,
written as $l|_n$ for a list of guarded strings $l$.

The denotation of $p^*$ uses the fact that $p^*$ and $1 + p;p^*$ are equivalent,
and decrements the size threshold on the recursive denotation of $p^*$ by $i$,
where $i$ is the length of the longest (nonzero) string in the denotation of
$p$, making sure to filter out guarded strings that are too long.


%% To convert from a list of guarded strings $l$ to a set of guarded
%% strings, we simply write $\lceil l\rceil$. Note that for any $p$,
%% $\varphi$, $\mathcal E$, and $n$, $\left\lceil\llparenthesis p
%% \rrparenthesis^{\varphi,\mathcal E}_n\right\rceil = \{g \mid g \in
%% \sem{ p }^{\varphi,\mathcal E}, |g| \leq n\}$.

\newcommand{\semlst}[2]{\llparenthesis#1\rrparenthesis^{\mathsf B, \varphi, \zeta}_{#2}}
\newcommand{\setToList}[1]{\lfloor#1\rfloor}

\begin{figure}[!t]
  \fbox{
    \begin{minipage}{0.95\columnwidth}
      \hspace{-0.9em}\(\begin{array}{r >{\triangleq}c >{}l}
      \semlst p 0 && [] \\
      \semlst 0 n && [] \\
      \semlst 1 n && \setToList \AphiB \\
      \semlst e n && [ \alpha e \beta \mid \alpha \mathrel{\sem{\zeta_e}^\varphi} \beta ] \\
      \semlst b n && \setToList{ \mathcal A_{\mathsf B}^{b\psi}} \\
      \semlst{p\! + \!q} n && \semlst p n \mathbin{\texttt{+\!+}} \semlst q n \\
      \semlst{p;q} n &&
      (\semlst p n \diamond \semlst q n)|_n\\
      \semlst{p^*} n && [] + (\semlst p n \diamond \semlst{p^*} {n-i}) |_n \\
      \multicolumn 3 r {\scriptstyle \text{where~}
          i \mathbin{=} \max\{1,\min\{|g|\mid g \in \semlst p n \}\}} \\
      \semlst{\neg p} n && \semlst {\bullet^*} n \setminus \semlst p n \\
      \semlst{\Diamond_ap } n &&
             [g' \mid g'\,\hat{R_a}\,g,\texttt{\footnotesize for}\,g\,\texttt{\footnotesize in}\,\semlst p n]
      \end{array}
      \)
    \end{minipage}}
  \caption{Bounded interpretation using lazy lists}
  \label{fig:bound-interp}
\end{figure}

%% The lists we use here are \emph{lazy} (as opposed to \emph{strict}), which
%% broadly means that computation is delayed until the value is needed. This allows
%% us to avoid computing large, unnecessary iterations.

\section{Syntax-semantics interface}
\label{sec:categorial-grammar}

\newcommand{\ebackt}{(e \backslash t)}
\begin{figure}[!t]
  \vspace{1em}
  \centering
  \footnotesize
  %% \begin{tabular}[t]{lcc}
  %%   \toprule
  %%       \textsc{Item} & \textsc{Type} & \textsc{Semantics}\\ \midrule
  %%       Amy & $e$ & $\hat{R}_a$\\
  %%       Bob & $e$ & $\hat{R}_b$\\
  %%       it & $d$ & $d$ \\
  %%       heads & $d \backslash_D t$ & $\lambda x.\bullet^*h$ \\
  %%       tails & $d \backslash_D t$ & $\lambda x.\bullet^*t$ \\
  %%       is & $(d \backslash t)\slash(d \backslash_D t)$ & $\lambda P~x.~P~x$ \\
  %%       knows & $(e \backslash t)\slash_M t$ &  $\lambda p~R.~p\vee{\Diamond}_Rp$\\
  %%       believes & $(e \backslash t)\slash_M t$ &  $\lambda p~R.~{\Diamond}_Rp$\\
  %%       that  & $(((e \backslash t){\slash_M}t) \backslash (e \backslash t)) \slash t$
  %%       & $\lambda p\,m\,R.{\neg}(m\,({\neg}p)\,R)$ \\
  %%       whether  & $(((e \backslash t){\slash_M} t) \backslash (e \backslash t)) \slash t$
  %%       & $\lambda p\,m\,R.{\neg}(m\,({\neg}p)\,R)$ \\
  %%       && \multicolumn{1}{r}{$+ {\neg}(m\,p\,R)$} \\ \bottomrule
  %% \end{tabular}

  \newcommand{\hdrspc}{0.3em}
    \begin{tabular}[t]{l>{\hspace{-.8em}}c >{\hspace{-.6em}}c}
    \toprule
    \textsc{Item} & \textsc{Type} & \textsc{Semantics}\\ \midrule
    \multicolumn{3}{c}{{\bf Statives with expletive subject}} \\[\hdrspc]
        heads & $d \backslash_D t$ & $\lambda x.\,\bullet^*h$ \\
        tails & $d \backslash_D t$ & $\lambda x.\,\bullet^*t$ \\ \midrule
    \multicolumn{3}{c}{{\bf Agents}} \\
        Amy & $e$ & $\hat{R}_a$\\
        Bob & $e$ & $\hat{R}_b$\\ \midrule
    \multicolumn{3}{c}{{\bf Auxiliary verbs and expletive subjects}} \\[\hdrspc]
        it & $d$ & {\em{dummy}} \\
        is & $(d \backslash t)\slash(d \backslash_D t)$ & $\lambda P\,x.\,P\,x$ \\
        it's & $(t \slash(d \backslash_D t)$ & $\lambda P\,x.\,P\,x$ \\
        isn't & $t\slash(d \backslash_D t)$ & $\lambda P\,x.\,\neg P\,x$ \\
        doesn't & $(e \backslash t)\slash(d \backslash_V t)$ & $\lambda P\,x.\,\neg P\,x$ \\  \midrule
    \multicolumn{3}{c}{{\bf Tensed attitude verbs}} \\[\hdrspc]
        knows & $(e \backslash t)\slash_M t$ &  $\lambda p\,R.\,p + {\Diamond}_Rp$\\
        believes & $(e \backslash t)\slash_M t$ &  $\lambda p\,R.\,{\Diamond}_Rp$\\ \midrule
    \multicolumn{3}{c}{{\bf Base form attitude verbs}} \\[\hdrspc]
        know & $(e \backslash_V t)\slash_M t$ &  $\lambda p\,R.\,p + {\Diamond}_Rp$\\
        believe & $(e \backslash_V t)\slash_M t$ &  $\lambda p\,R.\,{\Diamond}_Rp$\\ \midrule
    \multicolumn{3}{c}{{\bf Complementizers}} \\[\hdrspc]
        that  & $(((e \backslash t){\slash_M}t) \backslash (e \backslash t)) \slash t$
        & $\begin{pmatrix*}[l]
          \lambda p\,m\,R.\\
          \,\,{\neg}(m\,({\neg}p)\,R)
        \end{pmatrix*}$ \\[1.1em]
        whether  & $(((e \backslash t){\slash_M} t) \backslash (e \backslash t)) \slash t$
        & $\begin{pmatrix*}[l]
          \lambda p\,m\,R.\\
                  \,\, {\neg}(m\,({\neg}p)\,R) \\
          \,\,\,\, +\,{\neg}(m\,p\,R)
        \end{pmatrix*}$\\ \midrule
    \multicolumn{3}{c}{{\bf Complementizers for base form verbs}} \\[\hdrspc]
        that  & $(((e \backslash_V t){\slash_M}t) \backslash (e \backslash_V t)) \slash t$
        & $\begin{pmatrix*}[l]
          \lambda p\,m\,R. \\
          \,\, {\neg}(m\,({\neg}p)\,R)
        \end{pmatrix*}$ \\[1.1em]
        whether & $(((e \backslash_V t){\slash_M} t) \backslash (e \backslash_V t)) \slash t$
        & $\begin{pmatrix*}[l]
          \lambda p\,m\,R.\\
                  \,\, {\neg}(m\,({\neg}p)\,R) \\
          \,\,\,\, +\,{\neg}(m\,p\,R)
        \end{pmatrix*}$ \\ \midrule
        \multicolumn{3}{c}{{\bf Conjunction}} \\[\hdrspc]
        and & $(t \backslash t)\slash t$ &  $\lambda p\,q.\,p \wedge q$\\
        or & $(t \backslash t)\slash t$ &  $\lambda p\,q.\,p + q$\\
        and & $((\ebackt \backslash \ebackt)\slash \ebackt$ &  $\lambda p\,q\,x.\,p(x) \wedge q(x)$\\
        or & $((\ebackt \backslash \ebackt)\slash \ebackt$ &  $\lambda p\,q\,x.\,p(x) + q(x)$\\
        \bottomrule
      \end{tabular}%%
    \caption{Categorial grammar lexicon, showing word form (column 1), a
      categorial type (column 2), and a semantic translation in Epi\textsc{Kat} extended
      with lambda abstraction (column 3).}
  \label{lexiconfigure}
\end{figure}

English sentences are mapped to terms in the logical language via a semantically
interpreted multi-modal categorial grammar, consisting of a lexicon of words,
their categorial types, and interpretations in a logical lambda language. The
grammar covers basic statives ({\em it's heads}), {\em that-} and {\em
  whether}-complements, negation, and predicate and sentence
conjunction. Figure \ref{lexiconfigure} gives
the lexicon.\footnote{Category symbols use Lambek/Bar-Hillel notation for
  slashes, so that $(d \backslash t)\slash(d \backslash_D t)$ combines with $d
  \backslash_D t$ on the right to give a value that combines with $d$ on the
  left to give $t$. In the semantics, lambda abstractions with multiple
  parameters are written $\lambda x~y.~e$ rather than $\lambda x.\lambda y.e$.
  $d$ is the category of expletive {\em it}.} The online supplement includes a
version of this grammar that can be run using Barker and Shan's chart parser for
categorial grammar \citep{barker2005parser}. The grammar and semantics are
optimized for a simple fragment of English concerned with clausal
complementation. The agent names {\em Amy} and {\em Bob} contribute the
epistemic alternative relations for those agents, rather than individuals. The
root verb {\em believe} contributes existential modal force. The complementizers
{\em that} and {\em whether} are the heads of their dominating clauses, and
assemble an alternative relation, modal force, and proposition contributed by
the complement. These complementizers introduce the dual via two negations, in
order to express universal modal force.



  
Multimodal categories such as $\backslash_D$ and $\backslash_M$ are
used to control the derivation---phrases with these top-level slashes
can only combine syntactically as arguments.
%
%For instance the category of {\em
%  heads} is $d \backslash_D t$.  The dummy expletive subject {\em it}
%has category $d$.  However the phrase {\em it heads} of category $t$
%can not be formed, because $\backslash_D$ is not syntactically active
%as a function.  Instead {\em it is heads} is formed with a predicator
%{\em is} of category $(d \backslash t)\slash(d \backslash_D t)$.
%Similarly {\em knows} has a category with the top-level slash
%$\slash\!_M$, and combines to form a sentence as an argument of {\em
%  that} or {\em whether}. These complementizers have a category that
%looks for the category of {\em know} on the left, after combining with
%a complement sentence on the right.
The semantic translations in the third column of Figure \ref{lexiconfigure} use
the logical language, incremented with lambda.  The body of $\lambda
x.\bullet^*h$, which is the semantic lexical entry for {\em heads}, is a
term denoting the set of all worlds where the coin is heads, expressed as the
set of all guarded strings that end with a Boolean valuation where the primitive
proposition $h$ (it's heads) is true.
%There is $\lambda x$ at the
%front because the grammar formalism enforces a strict correspondence
%between syntactic and semantic types. However, the lambda does not
%bind anything, because sentences such as {\em it isn't heads} have an
%expletive subject.
The body of $\lambda p.\lambda R. {\Diamond}_R p$, which is the semantic lexical
entry of {\em believes}, is an term denoting the pre-image of the world-alternative
relation contributed by the subject.  This is not the right semantics for {\em Amy
  believes that it's heads}, because it has an existential modality ${\Diamond_R}p$,
rather than an universal modality ${\Box_R}p$.  This is corrected by the
complementizer {\em that}, which introduces the dual.

Sentences are parsed with a chart parser for categorial grammar.  The semantics for
complex phrases are obtained by application of semantic translations, accompanied by
beta reductions that eliminate all lambdas in logical forms for clauses.  In
consequence, the semantic term translating a sentence is an Epi\textsc{Kat} logical term.  Such
a term designates a set of possible words (guarded strings).  By way of example,
\exi{three}{a} is an English sentence with conjunction and several levels of clausal
embedding.  Using the grammar and parser, the sentence is mapped to the term in
\exi{three}{b}. \exi{three}{c} shows a simplified logical form constructed from
\exi{three}{b} using syntactic equivalences. Either term is compiled in an
implementation of the finite state calculus to a finite state machine with 10 states
and 23 edges, which accepts a countably infinite set of worlds.\footnote{ Machine
  sizes need not be small, especially as the cardinality of $\sf{B}$ increases.  A
  certain Epi\textsc{Kat} model with fourteen primitive Booleans has the set of worlds
  representented by a finite state machine with 184,794 states and 257,881 edges.}  In
this way the methodology ``directly'' represents the set of worlds denoted by
\exi{three}{a}.

\newcommand{\HEADS}{\bullet^*h}
\newcommand{\TAILS}{\bullet^*t}

\begin{examples} \label{three}
  \itema Amy knows that it's tails and doesn't believe that it's heads and
  believes that Bob believes that it's heads.
\itemb \(
\begin{array}[t]{ll}
  \neg(\neg \TAILS + \Diamond_a \neg \TAILS) & \wedge \\
  \neg \neg \Diamond_a \neg \HEADS & \wedge \\
  \neg \Diamond_a \neg\neg\Diamond_b \neg \HEADS
\end{array}  
\)
\itemc \(\K_{a}\!\TAILS \; \wedge \, \neg\B_{a}\!\HEADS \; \wedge \; \B_{a}\B_{b}\HEADS\)
\itemd Bob believes that it's tails.
\iteme \(\B_{b} \TAILS \)
\end{examples}


% This conversion was done by hand. It would be good to put it into the scheme program.
% edge      : 695 Amy knows that its tails and doesnt believe that its heads and believes that Bob believes that its heads        (0 19)  t
%semantics : (((Not ((Not tails) + (Diamond amy (Not tails)))) & (Not (Not (Diamond amy (Not heads))))) & (Not (Diamond amy (Not (Not (Diamond bob (Not heads)))))))
% (((Not ((Not tails) + (Diamond amy (Not tails))))
% & (Not (Not (Diamond amy (Not heads)))))
% & (Not (Diamond amy (Not (Not (Diamond bob (Not heads)))))))
%\[
%\begin{array}{ll}
%  \neg(\neg \TAILS + \Diamond_a \neg \TAILS) & \wedge \\
%  \neg \neg \Diamond_a \neg \HEADS & \wedge \\
%  \neg \Diamond_a \neg\neg\Diamond_b \neg \HEADS
%\end{array}  
%\]

Sentence \exi{three}{d} is assigned a logical form that is syntactically
equivalent to \exi{three}{e}. Logical relations between propositions are checked in
the finite state calculus by checking set-theoretic relations between
sets of worlds.  Propositional entailment from $p$ to $q$ is decided
by checking in an interpreter for the finite state calculus whether $p
- q$ is non-empty.  In the model defined by Figure 2, proposition
\exi{three}{c} entails proposition \exi{three}{e}.

Another way of using the Fst interpreter is to display worlds in a given proposition.
\ex{p4worlds} shows the three words of length three in proposition \exi{three}{c}.
In the first one, Bob looks at heads, then Amy looks at heads, then Amy secretly
turns the coin, believing that she is turning it from heads to tails because she
believed after two steps that the coin was heads. Intuitively sentence
\exi{three}{a} is true in this scenario.

\begin{example} \label{p4worlds}
\(
\begin{array}[t]{l}  
1 0 \; \bpH \; 1 0 \; \apH \; 1 0 \; \atHT \;  0 1 \\
1 0 \; \bpH \; 1 0 \; \atHT \; 0 1 \; \apT \; 0 1 \\
1 0 \; \apH \; 1 0 \; \bpH \; 1 0 \; \atHT \; 0 1 \\
\end{array}
\)
\end{example}  



\section{Discussion}
Epi\textsc{Kat} is designed for research in linguistic semantics, for
computational linguistics research on model-theoretically grounded semantics,
and for course modules on intensional formal semantics.
Currently, the grammar covers statives, {\em that-} and {\em
  whether}-complements, negation, and conjunction.
There are straightforward extensions to additional linguistic phenomena, such as
tense and perfective aspect \exi{extension}{a}, and the combination of
metaphysical modality and prospective aspect \exi{extension}{b}.

\begin{examples}   \label{extension}
  \itema Amy has learned that Bob had learned that it's heads.
  \itemb Amy might learn that it's heads.
\end{examples}  

The model framework is a constructive branching-time framework with metaphysical
and epistemic modalities, which will be applicable in linguistic semantic
research on combinations of tense, metaphysical modality, and epistemic
complementation \citep{thomason1984combinations, abusch1998generalizing,
  condoravdi2002temporal}. Connections with research on temporal constitution of
events remain to be explored
\citep{fernando2004finite,fernando2007observing,carlson2009tense}.

The syntactic part of the grammar formalism uses a standard approach to
categorial grammar \citep{steedman2000syntactic,bozsahin2012combinatory} to map
between strings (or trees or derivations) and logical terms, and is rigorous
enough to admit computational implementations
\citep{barker2005parser,bozsahin2021ccglab}.\footnote{Our grammar uses basic
  categorial grammar, not additional features such as combinators, type raising,
  continuations, or multimodal slashes that interact with the grammar in
  non-trivial ways. These features become relevant in more sophisticated
  grammars. Multimodal categories are used for phrases such as a tenseless verb
  phrase that semantically are functional, but combine in the grammar only as
  arguments.} While this approach is simple and attractive for our applications,
it would be possible to use Epi\textsc{Kat} with other frameworks that support lambda
extensions of logical languages.



%\newcommand{\Prime}[1]{{\rm\bf{#1}}}

The semantics presented here is exclusively concerned with events and worlds.
Semantic type systems for natural language usually include a type for
individuals \citep{gallin1975intensional}. In the coin example, the individuals
are hidden in the primitive event symbols such as $\apH$ and $\bpH$. This
situation would get worse in a more elaborate model with more agents and
multiple coins. The solution to this should be to base the world construction on
grounded or parametric event terms such as
$\Prime{peek}(\Prime{a},\!\Prime{k},\!\Prime{h})$, for ``agent $\Prime{a}$ peeks
at heads-up coin $\Prime{k}$'', rather than atomic event symbols. This approach
is found in research on situation calculus \citep{reiter2001knowledge}. How to
incorporate it in the scheme for Epi\textsc{Kat} specifications and into the
computational parts of the proposal is a topic for future research. The
ramifications of quantification or abstraction over individuals is unknown.
\citet{rooth2017finite} develops an approach based on introducing markers for
witnesses for discourse referents in the construction of worlds.

The development here is concerned with defining concrete computable possible
worlds models, and applying them in natural language semantics. Logical and
computational characterizations of Epi\textsc{Kat}, such as sound and complete
axioms, coalgebras, and relations to other logics, warrant further investigation.

Source code, examples, and instructions for building and running Epi\textsc{Kat}
are available under an open-source license at \url{https://github.com/ericthewry/epikat}.

%\clearpage

\bibliographystyle{acl_natbib}
% \bibliographystyle{chicago}


\bibliography{fs}


%% \section*{Appendix}
%% Complete categorial grammar extending Figure \ref{lexiconfigure}.  The online supplement
%% includes a version of this grammar that can be run using Barker and Shan's
%% chart parser for categorial grammar \citep{barker2005parser}.

\end{document}


\begin{figure}[!t]
  \vspace{1em}
  \centering
  \footnotesize
  \begin{tabular}[t]{lcc}
    \toprule
    \textsc{Item} & \textsc{Type} & \textsc{Semantics}\\ \midrule
    \multicolumn{3}{c}{{\bf Statives with expletive subject}} \\ 
        heads & $d \backslash_D t$ & $\lambda x.\bullet^*h$ \\
        tails & $d \backslash_D t$ & $\lambda x.\bullet^*t$ \\ \hline
    \multicolumn{3}{c}{{\bf Agents}} \\ 
        Amy & $e$ & $\hat{R}_a$\\
        Bob & $e$ & $\hat{R}_b$\\ \hline
    \multicolumn{3}{c}{{\bf Auxiliary verbs and expletive subjects}} \\        
        it & $d$ & {\em{dummy}} \\
        is & $(d \backslash t)\slash(d \backslash_D t)$ & $\lambda Px.Px$ \\
        it's & $(t \slash(d \backslash_D t)$ & $\lambda Px.Px$ \\
        isn't & $t\slash(d \backslash_D t)$ & $\lambda Px.\neg P x$ \\  
        doesn't & $(e \backslash t)\slash(d \backslash_V t)$ & $\lambda Px.\neg Px$ \\  \hline        
    \multicolumn{3}{c}{{\bf Tensed attitude verbs}} \\        
        knows & $(e \backslash t)\slash_M t$ &  $\lambda pR.p\vee{\Diamond}_Rp$\\
        believes & $(e \backslash t)\slash_M t$ &  $\lambda pR.{\Diamond}_Rp$\\ \hline
    \multicolumn{3}{c}{{\bf Base form attitude verbs}} \\        
        know & $(e \backslash_V t)\slash_M t$ &  $\lambda pR.p\vee{\Diamond}_Rp$\\
        believe & $(e \backslash_V t)\slash_M t$ &  $\lambda pR.{\Diamond}_Rp$\\ \hline       
    \multicolumn{3}{c}{{\bf Complementizers}} \\        
        that  & $(((e \backslash t){\slash_M}t) \backslash (e \backslash t)) \slash t$
        & $\lambda p\,m\,R.{\neg}(m\,({\neg}p)\,R)$ \\
        whether  & $(((e \backslash t){\slash_M} t) \backslash (e \backslash t)) \slash t$ 
        & $\lambda p\,m\,R.{\neg}(m\,({\neg}p)\,R)$ \\
        && \multicolumn{1}{r}{$+ {\neg}(m\,p\,R)$} \\ \hline
    \multicolumn{3}{c}{{\bf Complementizers for base form verbs}} \\        
        that  & $(((e \backslash_V t){\slash_M}t) \backslash (e \backslash_V t)) \slash t$
        & $\lambda p\,m\,R.{\neg}(m\,({\neg}p)\,R)$ \\
        whether  & $(((e \backslash_V t){\slash_M} t) \backslash (e \backslash_V t)) \slash t$ 
        & $\lambda p\,m\,R.{\neg}(m\,({\neg}p)\,R)$ \\
        && \multicolumn{1}{r}{$+ {\neg}(m\,p\,R)$} \\ % \bottomrule    
      \end{tabular}
    \caption{Categorial grammar lexicon. The first column has a word form.
    the second column a categorial type, and third column a semantic translation
    in a language that extends the Epi\textsc{Kat} logical language with lambda.}
  \label{lexiconfigure}
\end{figure} 





Old version of grammar figure.

  \begin{tabular}[t]{lcc}
    \toprule
        \textsc{Item} & \textsc{Type} & \textsc{Semantics}\\ \midrule  
        Amy & $e$ & $\hat{R}_a$\\
        Bob & $e$ & $\hat{R}_b$\\
        it & $d$ & $d$ \\
        heads & $d \backslash_D t$ & $\lambda x.\bullet^*h$ \\
        tails & $d \backslash_D t$ & $\lambda x.\bullet^*t$ \\
        is & $(d \backslash t)\slash(d \backslash_D t)$ & $\lambda P~x.~P~x$ \\
        knows & $(e \backslash t)\slash_M t$ &  $\lambda p~R.~p\vee{\Diamond}_Rp$\\
        believes & $(e \backslash t)\slash_M t$ &  $\lambda p~R.~{\Diamond}_Rp$\\
        that  & $(((e \backslash t){\slash_M}t) \backslash (e \backslash t)) \slash t$
        & $\lambda p\,m\,R.{\neg}(m\,({\neg}p)\,R)$ \\
        whether  & $(((e \backslash t){\slash_M} t) \backslash (e \backslash t)) \slash t$ 
        & $\lambda p\,m\,R.{\neg}(m\,({\neg}p)\,R)$ \\
        && \multicolumn{1}{r}{$+ {\neg}(m\,p\,R)$} \\ \bottomrule
      \end{tabular}


\section{Credits}

This document has been adapted
by Steven Bethard, Ryan Cotterrell and Rui Yan
from the instructions for earlier ACL and NAACL proceedings, including those for 
ACL 2019 by Douwe Kiela and Ivan Vuli\'{c},
NAACL 2019 by Stephanie Lukin and Alla Roskovskaya, 
ACL 2018 by Shay Cohen, Kevin Gimpel, and Wei Lu, 
NAACL 2018 by Margaret Michell and Stephanie Lukin,
2017/2018 (NA)ACL bibtex suggestions from Jason Eisner,
ACL 2017 by Dan Gildea and Min-Yen Kan, 
NAACL 2017 by Margaret Mitchell, 
ACL 2012 by Maggie Li and Michael White, 
ACL 2010 by Jing-Shing Chang and Philipp Koehn, 
ACL 2008 by Johanna D. Moore, Simone Teufel, James Allan, and Sadaoki Furui, 
ACL 2005 by Hwee Tou Ng and Kemal Oflazer, 
ACL 2002 by Eugene Charniak and Dekang Lin, 
and earlier ACL and EACL formats written by several people, including
John Chen, Henry S. Thompson and Donald Walker.
Additional elements were taken from the formatting instructions of the \emph{International Joint Conference on Artificial Intelligence} and the \emph{Conference on Computer Vision and Pattern Recognition}.

\section{Introduction}

The following instructions are directed to authors of papers submitted to ACL 2020 or accepted for publication in its proceedings.
All authors are required to adhere to these specifications.
Authors are required to provide a Portable Document Format (PDF) version of their papers.
\textbf{The proceedings are designed for printing on A4 paper.}


\section{Electronically-available resources}

ACL provides this description and accompanying style files at
\begin{quote}
\url{http://acl2020.org/downloads/acl2020-templates.zip}
\end{quote}
We strongly recommend the use of these style files, which have been appropriately tailored for the ACL 2020 proceedings.

\paragraph{\LaTeX-specific details:}
The templates include the \LaTeX2e{} source (\texttt{\small acl2020.tex}),
the \LaTeX2e{} style file used to format it (\texttt{\small acl2020.sty}),
an ACL bibliography style (\texttt{\small acl\_natbib.bst}),
an example bibliography (\texttt{\small acl2020.bib}),
and the bibliography for the ACL Anthology (\texttt{\small anthology.bib}).


\section{Length of Submission}
\label{sec:length}

The conference accepts submissions of long papers and short papers.
Long papers may consist of up to eight (8) pages of content plus unlimited pages for references.
Upon acceptance, final versions of long papers will be given one additional page -- up to nine (9) pages of content plus unlimited pages for references -- so that reviewers' comments can be taken into account.
Short papers may consist of up to four (4) pages of content, plus unlimited pages for references.
Upon acceptance, short papers will be given five (5) pages in the proceedings and unlimited pages for references. 
For both long and short papers, all illustrations and tables that are part of the main text must be accommodated within these page limits, observing the formatting instructions given in the present document.
Papers that do not conform to the specified length and formatting requirements are subject to be rejected without review.

The conference encourages the submission of additional material that is relevant to the reviewers but not an integral part of the paper.
There are two such types of material: appendices, which can be read, and non-readable supplementary materials, often data or code.
Additional material must be submitted as separate files, and must adhere to the same anonymity guidelines as the main paper.
The paper must be self-contained: it is optional for reviewers to look at the supplementary material.
Papers should not refer, for further detail, to documents, code or data resources that are not available to the reviewers.
Refer to Appendices~\ref{sec:appendix} and \ref{sec:supplemental} for further information. 

Workshop chairs may have different rules for allowed length and whether supplemental material is welcome.
As always, the respective call for papers is the authoritative source.


\section{Anonymity}
As reviewing will be double-blind, papers submitted for review should not include any author information (such as names or affiliations). Furthermore, self-references that reveal the author's identity, \emph{e.g.},
\begin{quote}
We previously showed \citep{Gusfield:97} \ldots
\end{quote}
should be avoided. Instead, use citations such as 
\begin{quote}
\citet{Gusfield:97} previously showed\ldots
\end{quote}
Please do not use anonymous citations and do not include acknowledgements.
\textbf{Papers that do not conform to these requirements may be rejected without review.}

Any preliminary non-archival versions of submitted papers should be listed in the submission form but not in the review version of the paper.
Reviewers are generally aware that authors may present preliminary versions of their work in other venues, but will not be provided the list of previous presentations from the submission form.

Once a paper has been accepted to the conference, the camera-ready version of the paper should include the author's names and affiliations, and is allowed to use self-references.

\paragraph{\LaTeX-specific details:}
For an anonymized submission, ensure that {\small\verb|\aclfinalcopy|} at the top of this document is commented out, and that you have filled in the paper ID number (assigned during the submission process on softconf) where {\small\verb|***|} appears in the {\small\verb|\def\aclpaperid{***}|} definition at the top of this document.
For a camera-ready submission, ensure that {\small\verb|\aclfinalcopy|} at the top of this document is not commented out.


\section{Multiple Submission Policy}
Papers that have been or will be submitted to other meetings or publications must indicate this at submission time in the START submission form, and must be withdrawn from the other venues if accepted by ACL 2020. Authors of papers accepted for presentation at ACL 2020 must notify the program chairs by the camera-ready deadline as to whether the paper will be presented. We will not accept for publication or presentation the papers that overlap significantly in content or results with papers that will be (or have been) published elsewhere.

Authors submitting more than one paper to ACL 2020 must ensure that submissions do not overlap significantly (>25\%) with each other in content or results.



\section{Formatting Instructions}

Manuscripts must be in two-column format.
Exceptions to the two-column format include the title, authors' names and complete addresses, which must be centered at the top of the first page, and any full-width figures or tables (see the guidelines in Section~\ref{ssec:title-authors}).
\textbf{Type single-spaced.}
Start all pages directly under the top margin.
The manuscript should be printed single-sided and its length should not exceed the maximum page limit described in Section~\ref{sec:length}.
Pages should be numbered in the version submitted for review, but \textbf{pages should not be numbered in the camera-ready version}.

\paragraph{\LaTeX-specific details:}
The style files will generate page numbers when {\small\verb|\aclfinalcopy|} is commented out, and remove them otherwise.


\subsection{File Format}
\label{sect:pdf}

For the production of the electronic manuscript you must use Adobe's Portable Document Format (PDF).
Please make sure that your PDF file includes all the necessary fonts (especially tree diagrams, symbols, and fonts with Asian characters).
When you print or create the PDF file, there is usually an option in your printer setup to include none, all or just non-standard fonts.
Please make sure that you select the option of including ALL the fonts.
\textbf{Before sending it, test your PDF by printing it from a computer different from the one where it was created.}
Moreover, some word processors may generate very large PDF files, where each page is rendered as an image.
Such images may reproduce poorly.
In this case, try alternative ways to obtain the PDF.
One way on some systems is to install a driver for a postscript printer, send your document to the printer specifying ``Output to a file'', then convert the file to PDF.

It is of utmost importance to specify the \textbf{A4 format} (21 cm x 29.7 cm) when formatting the paper.
Print-outs of the PDF file on A4 paper should be identical to the hardcopy version.
If you cannot meet the above requirements about the production of your electronic submission, please contact the publication chairs as soon as possible.

\paragraph{\LaTeX-specific details:}
PDF files are usually produced from \LaTeX{} using the \texttt{\small pdflatex} command.
If your version of \LaTeX{} produces Postscript files, \texttt{\small ps2pdf} or \texttt{\small dvipdf} can convert these to PDF.
To ensure A4 format in \LaTeX, use the command {\small\verb|\special{papersize=210mm,297mm}|}
in the \LaTeX{} preamble (below the {\small\verb|\usepackage|} commands) and use \texttt{\small dvipdf} and/or \texttt{\small pdflatex}; or specify \texttt{\small -t a4} when working with \texttt{\small dvips}.

\subsection{Layout}
\label{ssec:layout}

Format manuscripts two columns to a page, in the manner these
instructions are formatted.
The exact dimensions for a page on A4 paper are:

\begin{itemize}
\item Left and right margins: 2.5 cm
\item Top margin: 2.5 cm
\item Bottom margin: 2.5 cm
\item Column width: 7.7 cm
\item Column height: 24.7 cm
\item Gap between columns: 0.6 cm
\end{itemize}

\noindent Papers should not be submitted on any other paper size.
If you cannot meet the above requirements about the production of your electronic submission, please contact the publication chairs above as soon as possible.

\subsection{Fonts}

For reasons of uniformity, Adobe's \textbf{Times Roman} font should be used.
If Times Roman is unavailable, you may use Times New Roman or \textbf{Computer Modern Roman}.

Table~\ref{font-table} specifies what font sizes and styles must be used for each type of text in the manuscript.

\begin{table}
\centering
\begin{tabular}{lrl}
\hline \textbf{Type of Text} & \textbf{Font Size} & \textbf{Style} \\ \hline
paper title & 15 pt & bold \\
author names & 12 pt & bold \\
author affiliation & 12 pt & \\
the word ``Abstract'' & 12 pt & bold \\
section titles & 12 pt & bold \\
subsection titles & 11 pt & bold \\
document text & 11 pt  &\\
captions & 10 pt & \\
abstract text & 10 pt & \\
bibliography & 10 pt & \\
footnotes & 9 pt & \\
\hline
\end{tabular}
\caption{\label{font-table} Font guide. }
\end{table}

\paragraph{\LaTeX-specific details:}
To use Times Roman in \LaTeX2e{}, put the following in the preamble:
\begin{quote}
\small
\begin{verbatim}
\usepackage{times}
\usepackage{latexsym}
\end{verbatim}
\end{quote}


\subsection{Ruler}
A printed ruler (line numbers in the left and right margins of the article) should be presented in the version submitted for review, so that reviewers may comment on particular lines in the paper without circumlocution.
The presence or absence of the ruler should not change the appearance of any other content on the page.
The camera ready copy should not contain a ruler.

\paragraph{Reviewers:}
note that the ruler measurements may not align well with lines in the paper -- this turns out to be very difficult to do well when the paper contains many figures and equations, and, when done, looks ugly.
In most cases one would expect that the approximate location will be adequate, although you can also use fractional references (\emph{e.g.}, this line ends at mark $295.5$).

\paragraph{\LaTeX-specific details:}
The style files will generate the ruler when {\small\verb|\aclfinalcopy|} is commented out, and remove it otherwise.

\subsection{Title and Authors}
\label{ssec:title-authors}

Center the title, author's name(s) and affiliation(s) across both columns.
Do not use footnotes for affiliations.
Place the title centered at the top of the first page, in a 15-point bold font.
Long titles should be typed on two lines without a blank line intervening.
Put the title 2.5 cm from the top of the page, followed by a blank line, then the author's names(s), and the affiliation on the following line.
Do not use only initials for given names (middle initials are allowed).
Do not format surnames in all capitals (\emph{e.g.}, use ``Mitchell'' not ``MITCHELL'').
Do not format title and section headings in all capitals except for proper names (such as ``BLEU'') that are
conventionally in all capitals.
The affiliation should contain the author's complete address, and if possible, an electronic mail address.

The title, author names and addresses should be completely identical to those entered to the electronical paper submission website in order to maintain the consistency of author information among all publications of the conference.
If they are different, the publication chairs may resolve the difference without consulting with you; so it is in your own interest to double-check that the information is consistent.

Start the body of the first page 7.5 cm from the top of the page.
\textbf{Even in the anonymous version of the paper, you should maintain space for names and addresses so that they will fit in the final (accepted) version.}


\subsection{Abstract}
Use two-column format when you begin the abstract.
Type the abstract at the beginning of the first column.
The width of the abstract text should be smaller than the
width of the columns for the text in the body of the paper by 0.6 cm on each side.
Center the word \textbf{Abstract} in a 12 point bold font above the body of the abstract.
The abstract should be a concise summary of the general thesis and conclusions of the paper.
It should be no longer than 200 words.
The abstract text should be in 10 point font.

\subsection{Text}
Begin typing the main body of the text immediately after the abstract, observing the two-column format as shown in the present document.

Indent 0.4 cm when starting a new paragraph.

\subsection{Sections}

Format section and subsection headings in the style shown on the present document.
Use numbered sections (Arabic numerals) to facilitate cross references.
Number subsections with the section number and the subsection number separated by a dot, in Arabic numerals.

\subsection{Footnotes}
Put footnotes at the bottom of the page and use 9 point font.
They may be numbered or referred to by asterisks or other symbols.\footnote{This is how a footnote should appear.}
Footnotes should be separated from the text by a line.\footnote{Note the line separating the footnotes from the text.}

\subsection{Graphics}

Place figures, tables, and photographs in the paper near where they are first discussed, rather than at the end, if possible.
Wide illustrations may run across both columns.
Color is allowed, but adhere to Section~\ref{ssec:accessibility}'s guidelines on accessibility.

\paragraph{Captions:}
Provide a caption for every illustration; number each one sequentially in the form:
``Figure 1. Caption of the Figure.''
``Table 1. Caption of the Table.''
Type the captions of the figures and tables below the body, using 10 point text.
Captions should be placed below illustrations.
Captions that are one line are centered (see Table~\ref{font-table}).
Captions longer than one line are left-aligned (see Table~\ref{tab:accents}).

\begin{table}
\centering
\begin{tabular}{lc}
\hline
\textbf{Command} & \textbf{Output}\\
\hline
\verb|{\"a}| & {\"a} \\
\verb|{\^e}| & {\^e} \\
\verb|{\`i}| & {\`i} \\ 
\verb|{\.I}| & {\.I} \\ 
\verb|{\o}| & {\o} \\
\verb|{\'u}| & {\'u}  \\ 
\verb|{\aa}| & {\aa}  \\\hline
\end{tabular}
\begin{tabular}{lc}
\hline
\textbf{Command} & \textbf{Output}\\
\hline
\verb|{\c c}| & {\c c} \\ 
\verb|{\u g}| & {\u g} \\ 
\verb|{\l}| & {\l} \\ 
\verb|{\~n}| & {\~n} \\ 
\verb|{\H o}| & {\H o} \\ 
\verb|{\v r}| & {\v r} \\ 
\verb|{\ss}| & {\ss} \\
\hline
\end{tabular}
\caption{Example commands for accented characters, to be used in, \emph{e.g.}, \BibTeX\ names.}\label{tab:accents}
\end{table}

\paragraph{\LaTeX-specific details:}
The style files are compatible with the caption and subcaption packages; do not add optional arguments.
\textbf{Do not override the default caption sizes.}


\subsection{Hyperlinks}
Within-document and external hyperlinks are indicated with Dark Blue text, Color Hex \#000099.

\subsection{Citations}
Citations within the text appear in parentheses as~\citep{Gusfield:97} or, if the author's name appears in the text itself, as \citet{Gusfield:97}.
Append lowercase letters to the year in cases of ambiguities.  
Treat double authors as in~\citep{Aho:72}, but write as in~\citep{Chandra:81} when more than two authors are involved. Collapse multiple citations as in~\citep{Gusfield:97,Aho:72}. 

Refrain from using full citations as sentence constituents.
Instead of
\begin{quote}
  ``\citep{Gusfield:97} showed that ...''
\end{quote}
write
\begin{quote}
``\citet{Gusfield:97} showed that ...''
\end{quote}

\begin{table*}
\centering
\begin{tabular}{lll}
\hline
\textbf{Output} & \textbf{natbib command} & \textbf{Old ACL-style command}\\
\hline
\citep{Gusfield:97} & \small\verb|\citep| & \small\verb|\cite| \\
\citealp{Gusfield:97} & \small\verb|\citealp| & no equivalent \\
\citet{Gusfield:97} & \small\verb|\citet| & \small\verb|\newcite| \\
\citeyearpar{Gusfield:97} & \small\verb|\citeyearpar| & \small\verb|\shortcite| \\
\hline
\end{tabular}
\caption{\label{citation-guide}
Citation commands supported by the style file.
The style is based on the natbib package and supports all natbib citation commands.
It also supports commands defined in previous ACL style files for compatibility.
}
\end{table*}

\paragraph{\LaTeX-specific details:}
Table~\ref{citation-guide} shows the syntax supported by the style files.
We encourage you to use the natbib styles.
You can use the command {\small\verb|\citet|} (cite in text) to get ``author (year)'' citations as in \citet{Gusfield:97}.
You can use the command {\small\verb|\citep|} (cite in parentheses) to get ``(author, year)'' citations as in \citep{Gusfield:97}.
You can use the command {\small\verb|\citealp|} (alternative cite without  parentheses) to get ``author year'' citations (which is useful for  using citations within parentheses, as in \citealp{Gusfield:97}).


\subsection{References}
Gather the full set of references together under the heading \textbf{References}; place the section before any Appendices. 
Arrange the references alphabetically by first author, rather than by order of occurrence in the text.

Provide as complete a citation as possible, using a consistent format, such as the one for \emph{Computational Linguistics\/} or the one in the  \emph{Publication Manual of the American 
Psychological Association\/}~\citep{APA:83}.
Use full names for authors, not just initials.

Submissions should accurately reference prior and related work, including code and data.
If a piece of prior work appeared in multiple venues, the version that appeared in a refereed, archival venue should be referenced.
If multiple versions of a piece of prior work exist, the one used by the authors should be referenced.
Authors should not rely on automated citation indices to provide accurate references for prior and related work.

The following text cites various types of articles so that the references section of the present document will include them.
\begin{itemize}
\item Example article in journal: \citep{Ando2005}.
\item Example article in proceedings, with location: \citep{borschinger-johnson-2011-particle}.
\item Example article in proceedings, without location: \citep{andrew2007scalable}.
\item Example arxiv paper: \citep{rasooli-tetrault-2015}. 
\end{itemize}


\paragraph{\LaTeX-specific details:}
The \LaTeX{} and Bib\TeX{} style files provided roughly follow the American Psychological Association format.
If your own bib file is named \texttt{\small acl2020.bib}, then placing the following before any appendices in your \LaTeX{}  file will generate the references section for you:
\begin{quote}\small
\verb|\bibliographystyle{acl_natbib}|\\
\verb|\bibliography{acl2020}|
\end{quote}

You can obtain the complete ACL Anthology as a Bib\TeX\ file from \url{https://aclweb.org/anthology/anthology.bib.gz}.
To include both the anthology and your own bib file, use the following instead of the above.
\begin{quote}\small
\verb|\bibliographystyle{acl_natbib}|\\
\verb|\bibliography{anthology,acl2020}|
\end{quote}


\subsection{Digital Object Identifiers}
As part of our work to make ACL materials more widely used and cited outside of our discipline, ACL has registered as a CrossRef member, as a registrant of Digital Object Identifiers (DOIs), the standard for registering permanent URNs for referencing scholarly materials.

All camera-ready references are required to contain the appropriate DOIs (or as a second resort, the hyperlinked ACL Anthology Identifier) to all cited works.
Appropriate records should be found for most materials in the current ACL Anthology at \url{http://aclanthology.info/}.
As examples, we cite \citep{goodman-etal-2016-noise} to show you how papers with a DOI will appear in the bibliography.
We cite \citep{harper-2014-learning} to show how papers without a DOI but with an ACL Anthology Identifier will appear in the bibliography.

\paragraph{\LaTeX-specific details:}
Please ensure that you use Bib\TeX\ records that contain DOI or URLs for any of the ACL materials that you reference.
If the Bib\TeX{} file contains DOI fields, the paper title in the references section will appear as a hyperlink to the DOI, using the hyperref \LaTeX{} package.


\subsection{Appendices}
Appendices, if any, directly follow the text and the
references (but only in the camera-ready; see Appendix~\ref{sec:appendix}).
Letter them in sequence and provide an informative title:
\textbf{Appendix A. Title of Appendix}.

\section{Accessibility}
\label{ssec:accessibility}

In an effort to accommodate people who are color-blind (as well as those printing to paper), grayscale readability is strongly encouraged.
Color is not forbidden, but authors should ensure that tables and figures do not rely solely on color to convey critical distinctions.
A simple criterion:
All curves and points in your figures should be clearly distinguishable without color.

\section{Translation of non-English Terms}

It is also advised to supplement non-English characters and terms with appropriate transliterations and/or translations since not all readers understand all such characters and terms.
Inline transliteration or translation can be represented in the order of:
\begin{center}
\begin{tabular}{c}
original-form \\
transliteration \\
``translation''
\end{tabular}
\end{center}

\section{\LaTeX{} Compilation Issues}
You may encounter the following error during compilation: 
\begin{quote}
{\small\verb|\pdfendlink|} ended up in different nesting level than {\small\verb|\pdfstartlink|}.
\end{quote}
This happens when \texttt{\small pdflatex} is used and a citation splits across a page boundary.
To fix this, the style file contains a patch consisting of two lines:
(1) {\small\verb|\RequirePackage{etoolbox}|} (line 455 in \texttt{\small acl2020.sty}), and
(2) A long line below (line 456 in \texttt{\small acl2020.sty}).

If you still encounter compilation issues even with the patch enabled, disable the patch by commenting the two lines, and then disable the \texttt{\small hyperref} package by loading the style file with the \texttt{\small nohyperref} option:

\noindent
{\small\verb|\usepackage[nohyperref]{acl2020}|}

\noindent
Then recompile, find the problematic citation, and rewrite the sentence containing the citation. (See, {\em e.g.}, \url{http://tug.org/errors.html})

\section*{Acknowledgments}

The acknowledgments should go immediately before the references. Do not number the acknowledgments section.
Do not include this section when submitting your paper for review.

\bibliography{anthology,acl2020}
\bibliographystyle{acl_natbib}

\appendix

\section{Appendices}
\label{sec:appendix}
Appendices are material that can be read, and include lemmas, formulas, proofs, and tables that are not critical to the reading and understanding of the paper. 
Appendices should be \textbf{uploaded as supplementary material} when submitting the paper for review.
Upon acceptance, the appendices come after the references, as shown here.

\paragraph{\LaTeX-specific details:}
Use {\small\verb|\appendix|} before any appendix section to switch the section numbering over to letters.


\section{Supplemental Material}
\label{sec:supplemental}
Submissions may include non-readable supplementary material used in the work and described in the paper.
Any accompanying software and/or data should include licenses and documentation of research review as appropriate.
Supplementary material may report preprocessing decisions, model parameters, and other details necessary for the replication of the experiments reported in the paper.
Seemingly small preprocessing decisions can sometimes make a large difference in performance, so it is crucial to record such decisions to precisely characterize state-of-the-art methods. 

Nonetheless, supplementary material should be supplementary (rather than central) to the paper.
\textbf{Submissions that misuse the supplementary material may be rejected without review.}
Supplementary material may include explanations or details of proofs or derivations that do not fit into the paper, lists of
features or feature templates, sample inputs and outputs for a system, pseudo-code or source code, and data.
(Source code and data should be separate uploads, rather than part of the paper).

The paper should not rely on the supplementary material: while the paper may refer to and cite the supplementary material and the supplementary material will be available to the reviewers, they will not be asked to review the supplementary material.

\end{document}
